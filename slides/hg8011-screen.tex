\PassOptionsToPackage{xetex}{xcolor}
\PassOptionsToPackage{xetex}{graphicx}
\documentclass[a4paper,landscape,headrule,footrule,xetex]{foils}

\input{headx.tex}
\newcommand{\tra}[1]{\textcolor{olive}{\textsf{#1}}}
\usepackage{multicol}
\usepackage{booktabs,subscript}
\newcommand{\DF}[1]{\parbox{.6\textwidth}{#1}}
\begin{document}
\makexeCJKinactive
\renewcommand{\avmvalfont}{\it}
\header{~}{Screening Sherlock Holmes}{\normalsize based on slides by Brian Bergen-Aurand}
%Film Studies and Literature
\maketitle


\myslide{Overview}
\MyLogo{Film Studies and Literature}
\begin{itemize}
\item How (Screen) Language
Conveys Meaning—
Christian Metz Screens
Sherlock Holmes
\item 
Sherlock Holmes Baffled—Signs and
Interpretations
\item 
The Language of Screen Theory
\item Film Language: A Semiotics of the Cinema
\item The Adventure of the Speckled Band 1 \& 2
\end{itemize}
\myslide{Sherlock Holmes Baffled}

\href{https://www.youtube.com/watch?v=HYN4QzX9-EM}{Sherlock Holmes Baffled}

\begin{itemize} 
\item directed by Arthur Marvin in 1905 (music added)
\end{itemize}

\myslide{Sherlock Holmes Baffled} 

\begin{tabular}{cc}
Signs & Interpretations \\
  \begin{minipage}{0.45\linewidth}
    \begin{itemize}
    \item Holmes alone
    \item Smoking a cigar
    \item Holmes unkempt
\item Holmes draws and fires a gun
\item Holmes is “baffled”
\item A crime against Holmes
\item Holmes in the \textit{domestic}
sphere
    \end{itemize}
  \end{minipage}
  &
  \begin{minipage}{0.45\linewidth}
  \end{minipage}
\end{tabular}
\newpage
\begin{itemize}
\item Films start in 1895 
  \begin{itemize}
  \item  standard viewing position (front)
  \item  charge admissions
  \end{itemize}
\item Early films had a fixed camera
  \begin{itemize}
  \item So drawing rooms scenes were common
  \end{itemize}
\item What does Holmes look like \task
%%% Cap, pipe, magnifying glass, cape
  \begin{itemize}
  \item What is different in \textit{Sherlock Holmes baffled}?
  \item What is in the center?
  \item What is he wearing?
  \item What is the big difference in the narrative?
  \item No Watson!
  \end{itemize}


\myslide{Cinematic Signs / Cinematic Language}
\begin{itemize}
\item Themes and ideas
\item Film and the other arts
\item Realism, anti-realism, and \textit{mise-en-scène}
\\ ``telling a story through the cinematography, stage design and storyboarding''
\item Composition and the image
\item Sound
\end{itemize}

\myslide{The Language of Screen Theory}
\begin{itemize}
\item What is cinema?
  \begin{itemize}
  \item Cinema is a language in the sense of a
semiotic system.
\item The system of every film is constructed on
the basis of codes that a filmmaker either
adopts, transforms, or works against.
\end{itemize}
\item How do we understand it?
  \begin{itemize}
  \item 
Semiotics (semiology) is the science of
signs and of the codes used to understand
them
\end{itemize}
\end{itemize}

\myslide{A Semiotics of the Cinema}
\MyLogo{Christian Metz}
\begin{quotation}
  
\end{quotation}
“Everything is present in film: hence the
obviousness of film, and hence also its opacity.
The clarification of present by absent units
occurs much less than in verbal language. The
relationships \textit{in praesentia} are so rich that
they render the strict organization of \textit{in
absentia} relationships superfluous and
difficult. A film is difficult to explain because it
is easy to understand. The image impresses
itself on us, blocking everything that is not
itself.”
\end{itemize}

% Christian Metz:
% A Semiotics of the Cinema
% Inspector Bourrel, from the French TV show
% Les cinq dernières minutes (1958-92) in
% Impersonal Enunciation (2016).
% Story and Discourse
% Diegetic and Extra-Diegetic Narration
% When the detective speaks about the case
% to characters / to the audience
\myslide{Reception: A Semiotics of Holmes}
\begin{itemize}
\item Analyze the stories as detective fiction
  \begin{itemize}
  \item Structure, logic, and nature of detection
    \begin{itemize}
    \item Genre, technique, narrative
      \begin{itemize}
      \item Story \& Discourse (Enunciation)
      \end{itemize}
    \end{itemize}
  \end{itemize}
\item Analyze the stories as cultural representation
  \begin{itemize}
  \item Social and historical implications and
    resonances
    \begin{itemize}
    \item Institutional structures and relationships
      \begin{itemize}
      \item Race, Class, Gender, Sexuality, Technology, Law
      \end{itemize}
    \end{itemize}
  \end{itemize}
\end{itemize}
\myslide{The Speckled Band}
\begin{itemize}
\item \href{https://www.youtube.com/watch?v=1pGT51t5xNE}{The Case of
    the Speckled
    Band (1/2)} (12min)
\item \href{https://www.youtube.com/watch?v=L83Ugq8o3i8}{The Case of
    the  Speckled
    Band (1/2)} (12min)
\item \textit{Sherlock Holmes and Doctor Watson} is a television
  series created by Sheldon Reynolds in 1979
\item  It starred Geoffrey Whitehead, Donald Pickering and Patrick
  Newell in the title roles of Sherlock Holmes, Doctor Watson and
  Inspector Lestrade respectively.
\item It was a joint Polish-English production with 24 episodes

\item How is the story different?
\item Why doesn’t Holmes kill the snake?
\item Nice use of music to build suspense, \dots{}
\end{itemize}



% %%% Five major areas
% \item Cinematic Language
%  \begin{itemize}
%   \item Themes and Ideas
%   \item Film and other arts
%   \item Realism, anti-realism and \textit{mise-en-scène}
%   \item Composition and the image
%   \item Sound
%   \end{itemize}
% \item The language of Screen Theory  %%% slides
%   \begin{itemize}
%   \item What is cinema?
%   \item How do we understand it?
%   \end{itemize}
% \item Film works through five mediums  
%   \begin{itemize}
%   \item Image
%   \item Dialogue
%   \item Print
%   \item Music
%   \item Ambient sound and noise
%   \end{itemize}
% \item Diegetic (on-screen) and Extra-diegetic (off-screen)
% \end{itemize}

% \include{schedule}

\myslide{Speckled Band}
\begin{itemize}
\item Analyze the stories as detective fiction
  \begin{itemize}
  \item A rethinking of the detective in general, of Holmes
in particular
\item The detective in relation to the villain, Holmes in
relation to the Count
\end{itemize}
\item Analyze the stories as cultural representation
  \begin{itemize}
  \item Women in a society in which they cannot own
property
\item Women in a society in which they can own
property
\end{itemize}
\end{itemize}

\myslide{Screen Language — Signs}
\MyLogo{Repetition, Difference, and Ambiguity}

\begin{itemize}
\item The Adventure of the Speckled Band
\item “Miss Stoner, you have not. You are screening your
stepfather.”
\item 
  \begin{itemize}
  \item Count: Mr. Holmes do you ever hunt?
  \item Holmes: In a manner of speaking, Count.
  \item Count: Yes. Of course.
  \end{itemize}
\item Holmes: “I suppose we’ll have to think of the case
of the speckled band as only partially successful.”
\end{itemize}

\myslide{Sherlock Holmes in the 22nd Century}

An animated television series from 1999--2001 in which Sherlock Holmes
is brought back to life in the 22nd century to deal with a Moriarty
clone. The series was nominated for a Daytime Emmy for Special Class Animated Program

\href{https://www.youtube.com/watch?v=d9Eiy5FvYO0}{The Adventure of
  the Dancing Men} (22 min)

\myslide{Compare / Contrast}
\begin{center}
  \begin{tabular}{cc}
    Readership & Viewership \\
    The reader’s surrogate & The viewers’ surrogates \\
    Holmes’s methodology & Holmes’s methodology\\
    Detection & Detection \\
    Making sense of things & Making sense of things
  \end{tabular}
\end{center}
% In print (since 1887)
% On film (since 1900)
% On television (since ~1951)
% In video games (since 1984)
% On the internet (since 2004)Gaming Holmes
% Readership, Viewership,
% Playership
% http://www.sherlockian-sherlock.com/free-full-online-
% games.phpGaming Holmes
% The player’s surrogate
% Holmes’s methodology
% Detection
% Making sense of thingsFrom Screening Holmes
% to Translating Holmes
% Screening Holmes  Essay rather than
% adaptation.
% Screening Holmes involves a site of contact
% filled with repetition, difference, and
% ambiguity.
% Screening Holmes invites us to revisit all
% the texts and contexts involved in this
% contact zone.
% Sherlock Holmes in Translation
% Thank you!
\myslide{Screening Sherlock Holmes}
\begin{itemize}
\item In print (since 1887)
\item On film (since 1900)
\item On television (since ~1951)
\item In video games (since 1984)
\item On the internet (since 2004)
\end{itemize}


\myslide{Bibliography}

  \begin{itemize}
  \item  Christian Metz (1974) \textit{Film Language: A Semiotics of
      the Cinema} [Essais sur la signification au cinéma], Oxford
    University Press, 1974
% translated by Michael Taylor. p. cm.
  \end{itemize}

\end{document}

%%% Local Variables: 
%%% coding: utf-8
%%% mode: latex
%%% TeX-PDF-mode: t
%%% TeX-engine: xetex
%%% End: 

