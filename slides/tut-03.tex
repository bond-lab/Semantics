\documentclass[a4paper]{article}

\title{Semantics: Tutorial Three}
\author{Francis Bond \url{<bond@ieee.org>}}
\date{}%2011-08-15}
\usepackage{fontenc}
\usepackage{polyglossia}
\setmainlanguage{english}
\setmainfont{TeX Gyre Pagella}
\setsansfont[Ligatures=TeX]{TeX Gyre Heros}
\usepackage{xeCJK}
\setCJKmainfont{Noto Sans CJK JP}
\setCJKsansfont{Noto Sans CJK JP}
\setCJKmonofont{Noto Sans CJK JP}
%\newcommand{\ans}[1]{\hfill{#1}}
%\newcommand{\ans}[1]{}
%%% dynamically add answers

\input{answers}

\usepackage{multicol}
%\Restriction{}
%\rightfooter{}
%\leftheader{}
%\rightheader{}
\usepackage{mygb4e}
\newcommand{\lex}[1]{\textbf{\textit{#1}}}
\newcommand{\lx}[1]{\textbf{\textit{#1}}}
\newcommand{\ix}{\ex\it}
\newcommand{\con}[1]{\textsc{#1}}
\usepackage{url}
\usepackage[normalem]{ulem}
\newcommand{\ul}[1]{\uline{#1}}
\newcommand{\txx}[1]{\textbf{#1}}

\begin{document}
\maketitle

\begin{enumerate}
 \item Although English does not mark \textbf{evidentiality} grammatically, it
  can be expressed in other ways.  Consider the following situation:
$S$ ``Kim bit Sandy''.  How could you express the following situations:
  \begin{exe}
    \ex You think $S$ is true, but have no evidence
    \abox{I think Kim bit Sandy}
    \ex You saw $S$ occur
      \abox{I saw Kim bite Sandy}
     \ex You saw a bite mark on Sandy, matching Kim's dental work
      \abox{I deduce Kim bit Sandy from the bitemark
        \\ It seems that Kim bit Sandy}
      \ex Someone told you $S$
      \abox{People say,  Kim bit Sandy
        \\ I heard that Kim bit Sandy}
    \ex You are Sandy, and you experienced $S$
     \abox{Kim bit me}
  \end{exe}
  \abox{Of course, there could be other ways, these are just some of the possibilities}
  Are any of these expressed grammatically in a language that you speak?
  \abox{Japanese has two special constructions for seems/heard
  \\[2ex] It seems that Kim bit Sandy
        \\ キムが\ サンディを\ 噛だ\ よう\ だ
        \\ \textit{kimu-ga sandei-wo kanda you-da}
        \\ kim-\textsc{nom} sandy-\textsc{acc} bit seem \textsc{cop}
        \\[2ex]  I heard that Kim bit Sandy
        \\ キムが\ サンディを\ 噛だ\ そう\ だ
        \\ \textit{kimu-ga sandei-wo kanda sou-da}
        \\ kim-\textsc{nom} sandy-\textsc{acc} bit heard \textsc{cop}}
  
\item Find examples of each of the semantic roles from the story you are annotating:
  \begin{itemize}
  \item agent
  \item patient
  \item theme
  \item experiencer
  \item beneficiary
  \item location
  \item source
  \item goal
  \item stimulus
  \item instrument
  \end{itemize}
  \abox{Try to get the students to give at least one example in each language}
\item Find examples of modality from the story you are annotating:
  \begin{itemize}
  \item Epistemic (knowledge)
    \begin{itemize}
    \item auxilliary
    \item main verb
    \item adverb
    \end{itemize}
  \item Deontic (permisison/obligation)
    \begin{itemize}
    \item auxilliary
    \item main verb
    \item adverb
    \end{itemize}
  \end{itemize}
  \abox{Try to get the students to give at least one example in each language}
\end{enumerate}

\end{document}
