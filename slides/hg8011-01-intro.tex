\PassOptionsToPackage{xetex}{xcolor}
\PassOptionsToPackage{xetex}{graphicx}
\documentclass[a4paper,landscape,headrule,footrule,xetex,25pt]{foils}

\input{headx.tex}

\begin{document}
\header{}{Introduction, Organization}{What does it mean to mean?}
\maketitle

%\include{schedule}


\myslide{Welcome!}

\begin{itemize}
\item In this course we will introduce you to the study of meaning
  \begin{itemize}
  \item How meaning is built up from words and phrases
  \item How meaning depends on context
  \end{itemize}
\item Using the Sherlock Homes stories and the War with the Newts
  \begin{itemize}
  \item I try to make this as enjoyable as possible
  \item You get to read  a great story
  \end{itemize}
\end{itemize}


\myslide{Overview of today}

\begin{itemize}
\item How this course is organized
\item What is semantics
\item Why should we be interested in semantics
\item Syllabus; Administrivia
\end{itemize}




\myslide{Textbook and Readings}

\begin{itemize}
\item No required text book and not much reading
\item \emp{EXCEPT you must read the story assigned}
\item If you want to know more about semantics I recommend
  \begin{itemize}
  \item Saeed, John (2009). \textit{Semantics}. 3rd Edition. Wiley-Blackwell. 
  \item Lyons, John (1977) \textit{Semantics}.  Cambridge University Press
  \end{itemize}
\item Between now and next week, I expect you to read the assigned story.
\end{itemize}

\myslide{Studying meaning}

\begin{itemize}
\item I will teach you about meaning
\item You will then try to analyze the use of words 
  \\ in the Sherlock Holmes stories
  \\ in the War with the newts
  \begin{itemize}
  \item Word Meaning (sense)
    \\   https://compling.upol.cz/ntumc/cgi-bin/showcorpus.cgi
    \\  \eng{to knock up}
  \item Word and Sentence Meaning (sentiment)
    \\  \eng{Julia and I had no great pleasure in our lives}
  \item Idioms and metaphors
    \\ \eng{to cross someone's path}
  \end{itemize}
\item You must do the three online projects, each is 4--6 hours work

\end{itemize}



% Other References

% Biber, D., S. Conrad \& R. Reppen, Corpus Linguistics: Investigating Language Structure and Use. Cambridge University Press, 1998.

% Kennedy, G. An Introduction to Corpus Linguistics. Longman, 1998.

% McEnery, Tony et al. Corpus-Based Language Studies: An Advanced Resource Book. Routledge, 2006.

% McEnery, Tony and Andrew Wilson Corpus Linguistics 2nd ed, Edinburgh UP, 2001

% Sinclair, John. Corpus Concordance Collocation. Oxford: Oxford UP, 1991



\section{Introduction to Semantics}

\myslide{What is Semantics}
\begin{itemize}
\item Very broadly, semantics is the study of meaning
  \begin{itemize}
  \item Word meaning
  \item Sentence meaning
  \end{itemize}
\item Why do we want to study meaning?
\item What kind of knowledge does it take for a speaker to produce language and for a hearer to comprehend language? 
\end{itemize}

\myslide{Layers of Linguistic Analysis}
\begin{enumerate}\addtolength{\itemsep}{-0.75ex}
\item Phonetics \& Phonology
\item Morphology
\item Syntax
\item \txx{Semantics}
\item \txx{Pragmatics}
\item Stylistics
\end{enumerate}
% Two theories
% \begin{itemize}
% \item Semantics is \txx{autonomous}, a separate module
% \item Semantics is \txx{integrated} with other knowledge, inseparable
%   \begin{itemize}
%   \item linguistic knowledge is inseparable from encyclopedic knowledge
%   \end{itemize}
% \end{itemize}

\myslide{Do people share a common conceptual system?}

\begin{itemize}
\item What is a \lex{high school}?
\item What color is \lex{blue}?
\item What does \lex{verb} mean?
\item What is  \lex{carrot cake}?
\end{itemize}

\newpage

Japanese traffic lights are green (as required by international
agreements).  However they are typically called 青い \jpn[blue]{aoi},
the same word as the color of the sky.  Historically this color
historically covered both green and blue ``grue'',
with 緑 \jpn[green]{midori} being a later addition.  For this reason,
the Japanese government decided in 1973 to change the color of the go
light to the bluest possible hue of green!


\href{https://www.japantimes.co.jp/life/2013/02/25/language/the-japanese-traffic-light-blues-stop-on-red-go-on-what/#.WRmAuuWGNPZ}{The Japanese traffic light blues: Stop on red, go on what?}
 

\myslide{Word Meaning and Sentence Meaning}

\begin{itemize}
\item We store information about words in our \txx{mental lexicon}
  \begin{itemize}
  \item It is still unclear what exactly a word is!
  \end{itemize}
\item Words can be combined to form an infinite number of expressions
  \begin{itemize}
  \item This building up of meaning is referred to as \txx{composition}
  \item If the meaning of the whole can be deduced from the parts then it is \txx{compositional}
  \end{itemize}
\end{itemize}

\myslide{Reference and Sense}

\begin{itemize}
\item Words \txx{refer} to things in the world (like \iz{unicorn}s)
\item The meaning of a word across different contexts is often referred to as its \txx{sense}
  \begin{itemize}
  \item Same word can refer to different things
    \begin{itemize}
    \item English: \eng{I put my money in the \ul{bank}}
    \item English: \eng{I fell asleep at the river \ul{bank}}
    \end{itemize}
  \item Same basic concept can have different boundaries
    \begin{itemize}
    \item French: \eng[sheep/mutton]{mouton}
    \item English: \eng{sheep} vs \eng{mutton}
      
    \item Japanese: \eng[dove/pigeon]{hato}
    \item English: \eng{dove} vs \eng{pigeon}
    \end{itemize}
  \end{itemize}
\end{itemize}




\myslide{Representing meaning}
\MyLogo{Also vector space, description, images, video, \ldots}
\begin{itemize}
\item One of our goals will be to represent meaning
\item There are various ways to do this
  \begin{itemize}
  \item Syntactic trees
  \item Logical forms
  \item Thesauri and Ontologies 
  \item Translation
  \item Paraphrasing
  \end{itemize}
Can you think of others?

\item At the end of this course you should be able to use these to
  describe many aspects of meaning
\end{itemize}



\myslide{Language is normally under-specified}
\MyLogo{There are many meanings}
\begin{center}
\large We get \blu{words}: \\[2ex]
    \Large \eng{I saw a kid with a cat.} \\[3ex]
We want \emp{meaning}:
\\  \includegraphics[width=0.3\textwidth]{pics/1.png}
\end{center}



\myslide{I saw a kid with a cat$_1$}
\MyLogo{Thanks to Eddy and Zina Pozen for the pictures}

\hspace{-3em}\begin{tabular}{ll}

  \includegraphics[width=0.5\textwidth]{pics/1.png}
&
  \begin{minipage}{0.45\textwidth}
    \vspace*{-8ex}
\begin{scriptsize}
 {%
 \leaf{\emph{I}}
 \branch{1}{NP}
 \leaf{\emph{saw}}
 \branch{1}{V:see}
 \leaf{\emph{a}}
 \branch{1}{DET}
 \leaf{\emph{kid}}
 \branch{1}{N}
 \leaf{\emph{with a cat}}
\branch{1}{PP[together]}
\branch{2}{\ibar{N}}
\branch{2}{NP}
 \branch{2}{VP}
 \branch{2}{S}
 \qobitree}
\end{scriptsize}
\\[3ex]
 \small \iz{see(I, kid: \textsc{past});  with(kid, cat)}
\\[1ex] \iz{see $\subset$ perceive}
\\ \iz{kid $\sim$ child}
\\ \iz{with $\subset$ together}
\end{minipage}

\end{tabular}

\myslide{I saw a kid with a cat$_2$}
\MyLogo{}
\hspace{-3em}\begin{tabular}{ll}
  \includegraphics[width=0.5\textwidth]{pics/2.png}
&
  \begin{minipage}{0.45\textwidth}
    \vspace*{-10ex}
\begin{scriptsize}
 {%
 \leaf{\emph{I}}
 \branch{1}{NP}
 \leaf{\emph{saw}}
 \branch{1}{V:see}
 \leaf{\emph{a}}
 \branch{1}{DET}
 \leaf{\emph{kid}}
 \branch{1}{N}
\branch{2}{NP}
 \leaf{\emph{with a cat}}
\branch{1}{PP[together]}
 \branch{3}{VP}
 \branch{2}{S}
 \qobitree}
\end{scriptsize}
\\[3ex]
 \small 
 \iz{see(I, kid: \textsc{past}) with(I, cat)}
\\[1ex] \iz{see $\subset$ perceive}
\\ \iz{kid $\sim$ child}
\\ \iz{with $\subset$ together}
\end{minipage}
\end{tabular}




\myslide{I saw a kid with a cat$_3$}
\hspace{-3em}\begin{tabular}{ll}
  \includegraphics[width=0.5\textwidth]{pics/3.png}

&
  \begin{minipage}{0.45\textwidth}
    \vspace*{-20ex}
\begin{scriptsize}
 {%
 \leaf{\emph{I}}
 \branch{1}{NP}
 \leaf{\emph{saw}}
 \branch{1}{V:saw}
 \leaf{\emph{a}}
 \branch{1}{DET}
 \leaf{\emph{kid}}
 \branch{1}{N}
 \leaf{\emph{with a cat}}
\branch{1}{PP[together]}
\branch{2}{\ibar{N}}
\branch{2}{NP}
 \branch{2}{VP}
 \branch{2}{S}
 \qobitree}
\end{scriptsize}
\\[3ex]
 \small 
\iz{saw(I, kid: \textsc{pres});  with(kid, cat)}
\\[1ex] \iz{saw $\subset$ cut}
\\ \iz{kid $\sim$ child}
\\ \iz{with $\subset$ together}
\end{minipage}
\end{tabular}



\myslide{I saw a kid with a cat$_4$}
\hspace{-3em}\begin{tabular}{ll}
  \includegraphics[width=0.5\textwidth]{pics/4.png}
&
  \begin{minipage}{0.45\textwidth}
    \vspace*{-20ex}
\begin{scriptsize}
 {%
 \leaf{\emph{I}}
 \branch{1}{NP}
 \leaf{\emph{saw}}
 \branch{1}{V:saw}
 \leaf{\emph{a}}
 \branch{1}{DET}
 \leaf{\makebox[1em]{\emph{kid} [goat]}}
 \branch{1}{N}
\branch{2}{NP}
 \leaf{\emph{with a cat}}
\branch{1}{PP[together]}
 \branch{3}{VP}
 \branch{2}{S}
 \qobitree}
\end{scriptsize}
\\[3ex]
 \small 
\iz{saw(I, kid: \textsc{present}) with(I, cat)}
\\[1ex] \iz{saw $\subset$ cut}
\\ \iz{kid $\sim$ young goat}
\\ \iz{with $\subset$ together}
\end{minipage}
\end{tabular}


\myslide{I saw a kid with a cat$_5$}
\hspace{-3em}\begin{tabular}{ll}
  \includegraphics[width=0.5\textwidth]{pics/5.png}
&
  \begin{minipage}{0.45\textwidth}
    \vspace*{-20ex}
\begin{scriptsize}
 {%
 \leaf{\emph{I}}
 \branch{1}{NP}
 \leaf{\emph{saw}}
 \branch{1}{V:see}
 \leaf{\emph{a}}
 \branch{1}{DET}
 \leaf{\emph{kid}}
 \branch{1}{N}
\branch{2}{NP}
 \leaf{\emph{with a cat}}
\branch{1}{PP[instrument]}
 \branch{3}{VP}
 \branch{2}{S}
 \qobitree}
\end{scriptsize}
\\[3ex]
 \small 
 \iz{see(I, kid: \textsc{past}) with(I, cat) }
\\[1ex] \iz{see $\subset$ perceive}
\\ \iz{kid $\sim$ child}
\\ \iz{with $\subset$ instrumental}
 \end{minipage}
\end{tabular}


% \myslide{People are good at understanding}
% \MyLogo{We do this too}
% \begin{itemize}
% \item The words only hint at the meaning
% \item Many words can mean more than one thing (\blu{ambiguity})
% \item How can we \blu{model} and \blu{resolve} ambiguity?
% \item Look at the text and try to annotate the meaning
%   \begin{center}
%     Very hard work
%   \end{center}
% %   \begin{itemize}
% %   \item Deduce implicit models
% %     \begin{itemize}
% %     \item bag of words, $n$-gram chunks, \ldots
% %     \end{itemize}
% %   \item Define explicit models
% %     \begin{itemize}
% %     \item Grammars, lexicons and thesauri
% %     \end{itemize}
% %   \end{itemize}
% %\item Then build statistical language models (machine learning)
% \end{itemize}

\myslide{We can also use translations}
%\MyLogo{Work smarter}
%\addtocounter{exx}{-13}
\begin{exe}
  \ex \glll 我 看到了 一个 抱着 猫 的 孩子 \\
  wǒ   kàndàole    yīgè   bàozhe  māo  de    háizi. \\
  I saw one holding cat 's child \\
  \trans I did see a child holding a cat

  \ex \glll 我 抱着 猫 看到了 一个 孩子 \\
  wǒ  bàozhe māo kàndàole  yīgè     háizi \\
  I holding cat saw one  child \\
 \trans I holding a cat did see a child

  \ex \glll 我 鋸锯  一个 孩子 和 他/她 的 猫 \\
wǒ jù  yīgè    háizi  hé  tā/tā   de māo \\
%wo3 ju4  yi1ge4    háizi  he2  ta1ta1   de ma1o\\
 I      saw    one    child   and     he/she  's cat\\
 \trans I saw  a child and their cat 

  \ex \glll 我 和 一只 猫 鋸锯 一只 小 山羊 \\
wǒ hē  yīzhǐ  māo jù yīzhǐ xiǎo  shānyáng  \\
%wo3 he1  yi1zhi3  ma1o ju4 yi1zhi3 xia3o  sha1nya2ng    \\
I and one cat saw one small goat \\

\trans I and a child saw a young goat
  \ex \glll 我 用 一只 猫 看到了 一个 孩子 \\
wǒ yòng yīzhǐ  māo kàndàole  yīgè  háizi \\
I use one cat saw one child \\
\trans Using a cat, I did see a child
\end{exe}

\bigskip
\begin{center} \large
  Your turn: try to paraphrase --- translate into English
  \\ aim to be unambiguous, even if slightly disfluent\task
\end{center}




\myslide{Administrivia}
\begin{description}\addtolength{\itemsep}{-5mm}
\item [Coordinator]  Francis \ul{Bond} 
  {\small \url{<bond@ieee.org>}
    \\ ~ \hfill !\url{<francis.bond@upol.cz>}}
\item Details will all be  online:
  \begin{center}
    \url{https://bond-lab.github.io/Semantics/}    
  \end{center}
\end{description}


\myslide{Extra Credit}

\begin{itemize}
\item If you submit a correction that gets accepted for one of the
  resources we use then it shows good mastery of the material
  \begin{itemize}
  \item you can get 1-5\% extra credit (depending on the size/difficulty)
\\ Mark $n \propto 10^{n-1}$ lines of code/documentation
  \item You can't go over 100\%
  \end{itemize}
\item A correction can involve
  \begin{itemize}
  \item fixing an error in transcription or annotation
    \begin{itemize}
    \item spelling error
    \item wrong sense
    \item error in the dictionary
    \end{itemize}
  \item making the documentation easy to read
  \item pointing out an error in a translation / finding a new translation
%  \item  fixing a bug in code
%  \item extending the code with new capabilities
  \end{itemize}

\end{itemize}



\myslide{Student Responsibilities}

By remaining in this class, the student agrees to:
\begin{enumerate}\addtolength{\itemsep}{-1ex}
\item  Make a good-faith effort to learn and enjoy the material.
\item  Read assigned texts and participate in class discussions and activities.
\item Submit assignments on time.
\item Attend class at all times, barring special circumstances.
\item Get help early: approach us when you first have trouble understanding a concept or homework problem rather than complaining about a lack of understanding afterward.
\item Treat other students with respect in all class-related activities, including on-line discussions.
\end{enumerate}
\myslide{Attendance}
\begin{enumerate}\addtolength{\itemsep}{-1ex}
\item You are expected to attend all classes.
\item Be on time - lateness is disruptive to your own and others' learning.
\item Valid reasons for missing class include the following:
\begin{enumerate}
\item A medical emergency (including mental health emergencies)
\item A family emergency (death, birth, natural disaster, etc).
\end{enumerate}
\item There will be significant material covered in class that is not in your readings.  You cannot expect to do well without coming to class.
\item If you miss a class, it is your responsibility to get the notes, any handouts you missed, schedule changes, etc. from a classmate.
\end{enumerate}

\myslide{Remediation and Academic Integrity}
\begin{enumerate}
\item No late work will be accepted, except in the case of a documented excuse.
\item For planned, justified, absences on class days or days on which assignments are due, advance notice must be provided.
\item Cheating will not be tolerated. Violations, including plagiarism, will be seriously dealt with, and could result in \textbf{a failing grade for the entire course}.
\item Refer to the University Honour Code
\item As always, use your common sense and conscience.
\end{enumerate}


\myslide{The winning strategy}

\begin{itemize}
\item Read the stories before class (and after again, if necessary)
\item Work together: make study groups
\item Tasks: Discuss as much as you want (but not project 1), annotate your own answers
\item Ask questions \ldots\ early and often!
\end{itemize}


% \emp{Next Week}	Theories of meaning and the meaning of words

% \myslide{Acknowledgments and References}
% \MyLogo{If I have seen further, it is by standing on the shoulders of giants (Isaac Newton)}
% \begin{itemize}
% \item Course design and slides inherit from Nala Lee's HG202 course,
%   back in the depths of time (2009).
% \item Thanks to Na-Rae Han for 
%   inspiration for the student policies (from  \textit{LING 2050 Special Topics in Linguistics: Corpus linguistics}, U Penn; adapted).
%   \item Further Reading: 
%   \begin{itemize}
%   \item  Shannon, C.E. (1948), "A Mathematical Theory of Communication", Bell System Technical Journal, 27, pp. 379–423 \& 623–656, July \& October, 1948. \url{http://cm.bell-labs.com/cm/ms/what/shannonday/shannon1948.pdf}
%   \end{itemize}
% \end{itemize}



\end{document}

%%% Local Variables: 
%%% coding: utf-8
%%% mode: latex
%%% TeX-PDF-mode: t
%%% TeX-engine: xetex
%%% End: 

