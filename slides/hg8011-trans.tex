\PassOptionsToPackage{xetex}{xcolor}
\PassOptionsToPackage{xetex}{graphicx}
\documentclass[a4paper,landscape,headrule,footrule,xetex]{foils}

\input{headx.tex}
\newcommand{\tra}[1]{\textcolor{olive}{\textsf{#1}}}
\usepackage{multicol}
\usepackage{booktabs,subscript}
\newcommand{\DF}[1]{\parbox{.6\textwidth}{#1}}
\begin{document}
%\makexeCJKinactive
\renewcommand{\avmvalfont}{\it}
\header{~}{Translating Sherlock Holmes}{\normalsize
  based on slides by Uganda Kwan}
%Film Studies and Literature
\maketitle


\myslide{Holmes and translation}

\begin{itemize}
\item What did Sir Arthur Conan Doyle think about translation ?
\item  How can/do we know?
\item He has a story, where the translator is a very heroic character:
  \\
  \textit{The Adventure of the Greek Interpreter} \sh{GREE}
  \\ 
\end{itemize}

\myslide{Mr Melas}
\begin{itemize}
\item This story introduces Mycroft, Sherlock's brother
\item He says of the interpreter
  \begin{quote}
    \textit{Mr. Melas is a Greek by extraction,
as I understand, and he is a remarkable
linguist. He earns his living partly as interpreter
in the law courts and partly by acting as guide
to any wealthy Orientals who may visit the
Northumberland Avenue hotels. I think I will
leave him to tell his very remarkable
experience in his own fashion. } \end{quote}
\end{itemize}

\myslide{The story}
\MyLogo{Spoilers}
\begin{itemize}
\item ``\textit{It seemed to me to be obvious that this
Greek girl had been carried off by the young
Englishman named Harold Latimer.}''
\item The Daily News,
` \textit{Anybody supplying any information to the
whereabouts of a Greek gentleman named Paul
Kratides, from Athens, who is unable to speak
English, will be rewarded. A similar reward paid to
any one giving information about a Greek lady
whose first name is Sophy. X 2473.}'
\item Sophy has a large fortune
\item Kratides visited her,
\item Kratides disappeared
  \begin{itemize}
  \item kidnapped by 2 villains
  \end{itemize}
\end{itemize}

\myslide{The interpretation}
\begin{itemize}
\item 2 villians who speak only English
\item Kratides who speaks only Greek
\item Melas who speaks both \raisebox{-25ex}{\includegraphics[width=0.45\textwidth]{pics/The-strand-magazine-1893-09-the-adventure-of-the-greek-interpreter-p301-illu.jpg}}
  
  \newpage
  \newcommand{\TRNS}[1]{\uwave{\textcolor{blue}{#1}}}
  \begin{itemize}
    \item[M] ` ``You can do no good by this obstinacy. \TRNS{Who are you?}''
\item[K] ` ``I care not. \TRNS{I am a stranger in London.}''
\item[M] ` ``Your fate will be upon your own head. \TRNS{How long have you been
     here?}''
\item[K] ` ``Let it be so. \TRNS{Three weeks.}''
\item[M] ` ``The property can never be yours. \TRNS{What ails you?}''
\item[K] ` ``It shall not go to villains. \TRNS{They are starving me.}''
\item[M] ` ``You shall go free if you sign. \TRNS{What house is this?}''
\item[K] ` ``I will never sign. \TRNS{I do not know.}''
\item[M] ` ``You are not doing her any service. \TRNS{What is your name?}''
\item[K] ` ``Let me hear her say so. \TRNS{Kratides.}''
\item[M] ` ``You shall see her if you sign.\TRNS{Where are you from?}''
\item[K] ` ``Then I shall never see her. \TRNS{Athens.}''
\end{itemize}
\item Shows the presence of the translator/ interpreter
\item Highlights the importance of translating
\item Clearly Conan Doyle thought highly of translators!
\end{itemize}

\myslide{Translating into Chinese}

\begin{itemize}
\item This story was translated into Chinese
\item  希腊舌人 \cmn{xīlà shérén} “Greek tongue-man”
\item  What does it mean?
  \begin{itemize}
  \item Early Western Han Period, after 206 BCE
  \item shérén (舌人, tongue-men) was the common name for the
    government officials in charge of communicating with the
    neighbouring tribes of the Zhou Dynasty.  Ref: “Martha
    P.Y. Cheung, “Early Discourse on Yi”
  \end{itemize}
\item Now normally translated as ``希腊译员'' \cmn{xīlà yìyuán}
  ``Greek Interpreter''

\end{itemize}




%%%
%%% Note 
%%%
% One last thing! In the slides you went through that day on translation, there is a slide about the use of tongues in Chinese. The translation for 七嘴八舌 is actually more accurately defined as 'a lively discussion with everyone talking at once' instead of confusing opinions. It is usually referenced to someone saying something and people having different opinions and discussing them.

% 七 = seven
% 嘴 = mouth
% 八 = eight
% 舌 = tongues

% The idea of seven mouths and eight tongues used in just 4 Chinese
% characters create a sense of lively chatter and discussion amongst a
% group. So although tongues are often used to describe someone in a
% more negative way, I think using 七嘴八舌 to prove the point in  this case may seem a little weird for some of us.


\myslide{SH in China}

\begin{itemize}
\item SH introduced into China in 1896 (光緒 二十二年)
  \begin{itemize}
  \item Shiwu Bao 《時務報》 Chinese Progress
  \item Serialized (non-fiction) magazine
  \end{itemize}
\item Chinese modernization was not uniform
  \begin{itemize}
  \item “Why Should We Learn from the Barbarians?”
  \item China resisting foreign knowledge 1840-1860
  \item  Partial absorption 1860-1890 (self strengthening movement)
  \item  Conscious learning 1895-
  \item  Wholesale westernization 1911-1917
  \end{itemize}
\end{itemize}

\myslide{True Crime}
\begin{itemize}\addtolength{\itemsep}{-1ex}
\item The first stories were:
  \begin{enumerate}
  \item 《英包探勘盗密约案》1896.9.27--10.27 (Vol 6-9) \sh{NAVA}
%    \\ 《The Adventure of the Naval Treaty》
  \item  《记伛者复仇事》1896.11.5-25 (Vol 10-12)  \sh{CROO} 
% 《The Adventure of the Crooked Man》
  \item  《继父诳女破案》 1897.4.22—5.12 (Vol 24-26) \sh{IDEN}
    % ,《A Case of Identity》1891.9
\item  《呵尔唔斯缉案被戕》 1897.5.22-6.20 Vol27-30) \sh{FINA}
%《The Adventure of the Final Problem》
\end{enumerate}
\item  Rewritten from first person to third person
  \begin{itemize}
  \item Matches the Chinese court case/ kung-an 公案
  \item common in late Imperial China
  \item this matches the readers' taste and social convention
  \end{itemize}
\item Subsequent translations treated the stories as fiction

\end{itemize}


\myslide{英包探勘盗密约案}

\begin{exe}
\ex   \glll 英 包探 勘盗 密 约案\\
  yīng bāotàn kāndào  mìyuē àn \\
  English detective steal secret agreement \\
  
 \trans The Adventure of the Naval Treaty
  \end{exe}

  \begin{itemize}
  \item Re-written in the style of the Gong'an or crime-case fiction
    (公案小说)  a subgenre of Chinese crime fiction where government
    magistrates solve criminal cases
  \item Famous examples are Justice Bao (包青天) 
    and Judge Dee (狄公案)
\includegraphics[width=0.2\textwidth]{pics/judgebao.jpeg}

\end{itemize}

\myslide{公案小说}

\begin{itemize}
\item Chinese Magistrate serves multiple capacities:
  \begin{itemize}
  \item investigator
  \item interrogator 
  \item prosecutor 
  \item judge  
  \item overseer of execution
  \end{itemize}
\item Western detective
  \begin{itemize}
  \item solves the mystery independently of the state
  \item justice may be  outside the state's command
  \item but in cooperation
\end{itemize}
\end{itemize}
\myslide{夏洛克·福爾摩斯}

\begin{itemize}
\item Xiàluòkè Fú'ěrmósī
\item Translations came in through Hokkien
  \begin{itemize}
  \item Sherlock Holmes 
  \item “歇洛克.呵尔唔斯” 1895-6
\\ Translated by Lin Shu (1852-1924) from Fu Jian province who spoke
Hokkien
\\ this would be pronounced  “ho-ee-mo-see” or “hock-ee-buah-see”
  \item  休洛克福而摩司 1902
  \item  福尔摩斯 1904-5
  \end{itemize}
\item Watson 滑震 ``slippery shaken!''
\end{itemize}

\myslide{Cheng, Xiaoqing (程 小青)}
\begin{itemize}
\item Cheng Xiaoqing (1893-1976) was the most popular author of
  Chinese detective fiction in the first half of the twentieth century
\item He translated Holmes stories into both classical Chinese and then
  vernacular
\item  He then went on to write his own crime series set in Shanghai
  \begin{itemize}
  \item Huo Sang ($\approx$ Sherlock)
  \item Bao Lang ($\approx$ Watson)
  \item The duo lived in a spacious apartment on Aiwen Road, where Huo
    Sang played the violin (badly) and smoked Golden Dragon cigarettes
    as he mulled over his cases.
  \end{itemize}
\item a 1940s edition of his Huo Sang stories ran to 30 volumes!
\end{itemize}


\myslide{Sherlock Holmes in Japan}
\MyLogo{\url{http://www.holmesjapan.jp/english/koba02.htm}}
\begin{itemize}
\item Sherlock Holmes reached Japan for the first time in
  1894:\footnote{3 years after it appeared} an
  abridged Japanese version of "The Man with the Twisted Lip" was
  published in the January issue of a magazine "Nippon-jin",
  translated from a French translation by Maurice LeBlanc.
\item  In April 1899, the newspaper "Mainichi Shinbun" began a
  three-month series featuring an adaptation (by unknown translator)
  of"A Study in Scarlet"
\item Between 12 July and 4 November of the same year, the "Chuo
  Shinbun" newspaper published the first complete Japanese translation
  of a Holmes story, ``The Adventures of Sherlock Holmes'', in a
  translation by Gaishi Nan-yo.
\item The first book-length translation appeared in 1907.
\end{itemize} 

\myslide{Sherlock Holmes in Japan}
\begin{itemize}
\item After that, Sherlock Holmes was always popular in Japan, but it
  was not until 1955, twenty-nine years after Conan Doyle finished his
  last Holmes stories, that the translation of all 60 stories by Ken
  Nobuhara appeared in Japan.
  
\item The Japan Sherlock Holmes Club was founded in 1977.
\item  Between 1907 and September 1997, there were 95
  separate translations of Sherlock Holmes stories in Japanese, 34 adaptations for children, 249 parodies
  and pastiches, and nearly 400 other pieces of Sherlockiana.
\item According to the survey of libraries in Junior High Schools in
  Japan, translated Sherlock Holmes stories are in the top five titles which are most often read or requested. 
\item  Sherlock Holmes is the most famous English person in Japan, and
  better known than either the Beatles or the late Princess Diana (who follow in the second and the third place).
\end{itemize}


\myslide{Sherlock (manga)}
\MyLogo{The BBC Sherlock as a manga!}
\includegraphics[width=0.4\textwidth]{pics/sherlock_manga_pink.jpg}
\includegraphics[width=0.4\textwidth]{pics/sherlock_manga_scandal.jpg}



\myslide{Miss Sherlock}

\includegraphics[width=0.4\textwidth]{pics/miss_sherlock.jpg}
\includegraphics[width=0.4 \textwidth]{pics/miss_sherlock_sherlock.jpg}

\begin{itemize}\addtolength{\itemsep}{-1ex}
\item Sara "Sherlock" Shelly Futaba  solving various mysteries in
  modern-day Tokyo.
\item Her assistant is Dr. Wato Tachibana (Wato-san!)
\item The main antagonist is Moriwaki, \ldots
\item  The first major series to cast a woman as Holmes-like detective
\item Inspired by the BBC's Sherlock
\end{itemize}


\myslide{Bibliography}

  \begin{itemize}
  \item Cheng Xiaoqing (2007) \textit{Sherlock in Shanghai: Stories of
      Crime and Detection} Tr. Timothy C. Wong. Honolulu: University
    of Hawai’i Pres,  ISBN 978-0-8248-3099-1 
  \item  Christian Metz (1974) \textit{Film Language: A Semiotics of
      the Cinema} [Essais sur la signification au cinéma], Oxford
    University Press, 1974
% translated by Michael Taylor. p. cm.
  \end{itemize}

\end{document}

%%% Local Variables: 
%%% coding: utf-8
%%% mode: latex
%%% TeX-PDF-mode: t
%%% TeX-engine: xetex
%%% End: 

