\PassOptionsToPackage{xetex}{xcolor}
\PassOptionsToPackage{xetex}{graphicx}
\documentclass[a4paper,landscape,headrule,footrule,xetex]{foils}

\input{headx.tex}
\newcommand{\tra}[1]{\textcolor{olive}{\textsf{#1}}}
\usepackage{multicol}
\usepackage{booktabs,subscript}
\newcommand{\DF}[1]{\parbox{.6\textwidth}{#1}}
\begin{document}
%\makexeCJKinactive
\renewcommand{\avmvalfont}{\it}
\header{~}{Sherlock Holmes in Translation}{\normalsize
  based on slides by Uganda Kwan}
%Film Studies and Literature
\maketitle

%%%
%%% Note 
%%%
% One last thing! In the slides you went through that day on translation, there is a slide about the use of tongues in Chinese. The translation for 七嘴八舌 is actually more accurately defined as 'a lively discussion with everyone talking at once' instead of confusing opinions. It is usually referenced to someone saying something and people having different opinions and discussing them.

% 七 = seven
% 嘴 = mouth
% 八 = eight
% 舌 = tongues

% The idea of seven mouths and eight tongues used in just 4 Chinese
% characters create a sense of lively chatter and discussion amongst a
% group. So although tongues are often used to describe someone in a
% more negative way, I think using 七嘴八舌 to prove the point in  this case may seem a little weird for some of us.


\myslide{SH in China}

\begin{itemize}
\item SH introduced into China in 1896 (光緒 二十二年)
  \begin{itemize}
  \item Shiwu Bao 《時務報》 Chinese Progress
  \item Serialized (non-fiction) magazine
  \end{itemize}
\item Chinese modernization was not uniform
  \begin{itemize}
  \item “Why Should We Learn from the Barbarians?”
  \item China resisting foreign knowledge 1840-1860
  \item  Partial absorption 1860-1890 (self strengthening movement)
  \item  Conscious learning 1895-
  \item  Wholesale westernization 1911-1917
  \end{itemize}
\end{itemize}

\myslide{True Crime}
\begin{itemize}\addtolength{\itemsep}{-1ex}
\item The first stories were:
  \begin{enumerate}
  \item 《英包探勘盗密约案》1896.9.27--10.27 (Vol 6-9) \sh{NAVA}
%    \\ 《The Adventure of the Naval Treaty》
  \item  《记伛者复仇事》1896.11.5-25 (Vol 10-12)  \sh{CROO} 
% 《The Adventure of the Crooked Man》
  \item  《继父诳女破案》 1897.4.22—5.12 (Vol 24-26) \sh{IDEN}
    % ,《A Case of Identity》1891.9
\item  《呵尔唔斯缉案被戕》 1897.5.22-6.20 Vol27-30) \sh{FINA}
%《The Adventure of the Final Problem》
\end{enumerate}
\item  Rewritten from first person to third person
  \begin{itemize}
  \item Matches the Chinese court case/ kung-an 公案
  \item common in late Imperial China
  \item this matches the readers' taste and social convention
  \end{itemize}
\item Subsequent translations treated the stories as fiction

\end{itemize}

\myslide{夏洛克·福爾摩斯}

\begin{itemize}
\item Xiàluòkè Fú'ěrmósī
\item Translations came in through Hokkien
  \begin{itemize}
  \item Sherlock Holmes 
  \item “歇洛克.呵尔唔斯” 1895-6
\\ Translated by Lin Shu (1852-1924) from Fu Jian province who spoke
Hokkien
\\ this would be pronounced  “ho-ee-mo-see” or “hock-ee-buah-see”
  \item  休洛克福而摩司 1902
  \item  福尔摩斯 1904-5
  \end{itemize}
\end{itemize}

\myslide{Cheng, Xiaoqing (程 小青)}
\begin{itemize}
\item Cheng Xiaoqing (1893-1976) was the most popular author of
  Chinese detective fiction in the first half of the twentieth century
\item He translated Holmes stories into both classical Chinese and then
  vernacular
\item  He then went on to write his own crime series set in Shanghai
  \begin{itemize}
  \item Huo Sang ($\approx$ Sherlock)
  \item Bao Lang ($\approx$ Watson)
  \item The duo lived in a spacious apartment on Aiwen Road, where Huo
    Sang played the violin (badly) and smoked Golden Dragon cigarettes
    as he mulled over his cases.
  \end{itemize}
\item a 1940s edition of his Huo Sang stories ran to 30 volumes!
\end{itemize}


\myslide{Sherlock Holmes in Japan}
\MyLogo{\url{http://www.holmesjapan.jp/english/koba02.htm}}
\begin{itemize}
\item Sherlock Holmes reached Japan for the first time in 1894: an
  abridged Japanese version of "The Man with the Twisted Lip" was
  published in the January issue of a magazine "Nippon-jin".
\item  In April 1899, the newspaper "Mainichi Shinbun" began a
  three-month series featuring an adaptation (by unknown translator)
  of"A Study in Scarlet"
\item Between 12 July and 4 November of the same year, the "Chuo
  Shinbun" newspaper published the first complete Japanese translation
  of a Holmes story, "The Adventures of Sherlock Holmes", in a
  translation by Gaishi Nan-yo.
\item The first book-length translation appeared in 1907. 
\item After that, Sherlock Holmes was always popular in Japan, but it was not until 1955, twenty-nine years after Conan Doyle finished his last Holmes stories, that the translation of all 60 stories by Ken Nobuhara appeared in Japan.
\end{itemize}

\myslide{Bibliography}

  \begin{itemize}
  \item Cheng Xiaoqing (2007) \textit{Sherlock in Shanghai: Stories of
      Crime and Detection} Tr. Timothy C. Wong. Honolulu: University
    of Hawai’i Pres,  ISBN 978-0-8248-3099-1 
  \item  Christian Metz (1974) \textit{Film Language: A Semiotics of
      the Cinema} [Essais sur la signification au cinéma], Oxford
    University Press, 1974
% translated by Michael Taylor. p. cm.
  \end{itemize}

\end{document}

%%% Local Variables: 
%%% coding: utf-8
%%% mode: latex
%%% TeX-PDF-mode: t
%%% TeX-engine: xetex
%%% End: 

