\PassOptionsToPackage{xetex}{xcolor}
\PassOptionsToPackage{xetex}{graphicx}
\documentclass[a4paper,landscape,headrule,footrule,xetex]{foils}

\input{headx.tex}
\newcommand{\tra}[1]{\textcolor{olive}{\textsf{#1}}}
\usepackage{multicol}
\usepackage{booktabs,subscript}
\newcommand{\DF}[1]{\parbox{.6\textwidth}{#1}}
\begin{document}
\makexeCJKinactive
\renewcommand{\avmvalfont}{\it}
\header{~}{Reading Sherlock Holmes II}{\normalsize based on slides by Jane Wong Y. C. }
\maketitle

%\include{schedule}

\myslide{The Adventure of the Dancing Men}
\MyLogo{The Victorian Web (See References)}
\begin{itemize}
\item Holmes, upholder of justice and drug addict?
\item In the Victorian era, “sale of opium, laudanum,
cocaine and morphine was legal. Victorian
users took these dangerous drugs as selfmedication and as recreation” (VW)
\item Holmes's recreational use of drugs can be
explained in two ways.
\end{itemize}



\myslide{From A Study in Scarlet (1887)}

\begin{quotation}
Nothing could exceed his energy when the working fit was upon him: but
now and again a reaction would seize him, and for days on end he would lie
upon the sofa in the sitting- room, hardly uttering a word or moving a
muscle from morning to night. On these occasions I have noticed such a
dreamy, vacant expression in his eyes, that I might have suspected him of
being addicted to the use of some narcotic, had not the temperance and
cleanliness of his whole life forbidden such a notion. 
(The Complete Sherlock Holmes, vol. I, 13)
\end{quotation}

\myslide{From The Sign of Four (1890)}
\begin{quotation}
Sherlock Holmes took his bottle from the corner of the mantel-piece and his
hypodermic syringe from its neat morocco case. With his long, white,
nervous fingers he adjusted the delicate needle, and rolled back his left
shirt-cuff. For some little time his eyes rested thoughtfully upon the sinewy
forearm and wrist all dotted and scarred with innumerable puncture-marks.
Finally, he thrust the sharp point home, pressed down the tiny piston, and
sank back into the velvet-lined arm-chair with a long sigh of satisfaction.
(The Complete Sherlock Holmes, vol. I, 99) [not unusual for heroes to have
fatal flaw—superman; more “relatable”)
\end{quotation}


\myslide{The Adventure of the Dancing Men}

\begin{itemize}
\item The introduction
of Holmes “his long, thin back curved over a
chemical vessel in which he was brewing a particularly malodorous
product.”
\item “If there is an afternoon train to town, Watson, I think we should do
well to take it, as I have a chemical analysis of some interest to
finish, and this investigation draws rapidly to a close.”
\item 
Holmes brilliant or because others around him lack the skills of
close observation?
\item 
The drawings of the Dancing Men.
\item Why are they not described in the narrative but illustrated?
\item Hilton Cubitt’s family background:
  \begin{quote}
“You'll just ask me anything that I don't make clear. I'll begin at the
time of my marriage last year; but I want to say first of all that,
though I'm not a rich man, my people have been at Ridling Thorpe for
a matter of five centuries, and there is no better-known family in the
county of Norfolk.”
\end{quote}
\end{itemize}
\myslide{The Adventure of the Dancing Men}
\begin{itemize}
\item “I am only
a simple Norfolk squire, but there is not a
man in England who ranks his family honour more
highly than I do.”
\item The significance of Elsie Patrick.
\item The marriage as irrational:
  \begin{quote}
You'll think it very mad, Mr Holmes, that a man of a
good old family should marry a wife in this fashion,
knowing nothing of her past or of her people \ldots
\end{quote}
\item
Holmes not simply a character that makes the
characters from prominent families look foolish, he
preserves their reputations.
\end{itemize}

\myslide{The Adventure of the Dancing Men}
\begin{itemize}
\item Legitimizing
Holmes (as non-official source of authority).
\item 
Representation of legal authority: Inspector Martin of the
Norfolk Constabulary.
\item 
“Then you must have important evidence of which we are
ignorant, for they were said to be a most united couple.”
\item 
The co-operation of police and detective.
\item 
Inspector Martin impressed by Holmes genius.
\item 

Authority undermined as Holmes orders the inspector
around.
\item 
The character traits of Inspector Martin.
\end{itemize}

\myslide{The Hound of the Baskervilles}


\begin{itemize}
\item The third (but most succesful) novel
\item serialised in The Strand Magazine from August 1901 to April 1902
\item the Holmes story since his apparent death in "The
  Final Problem" (although set before the final problem)
\item emphasizes the eerie setting and mysterious atmosphere of the
  moor
\\ \eng{ “The longer one stays here, the more does the spirit of the moor sink into one’s soul, its vastness, and also its grim charm.” }
\end{itemize}

\myslide{Some Classic Detective Fiction Tropes}

\begin{itemize}
\item Supernatural turns out to be natural
  \begin{itemize}
  \item just like in Scooby Doo
  \end{itemize}
\item Red herrings throw us off the scent
  \begin{itemize}
  \item the escaped convict
  \end{itemize}
\item Punishment comes from fate
\\ \eng{Somewhere in the heart of the great Grimpen Mire, down in the foul slime of the huge morass which had sucked him in, this cold and cruel-hearted man is forever buried.}
\end{itemize}

\myslide{The ungothic novel}

\begin{itemize}
\item 
As soon as Dr. Mortimer arrives to unveil the mysterious curse of the
Baskervilles, Hound wrestles with questions of natural and
supernatural occurrences. The doctor himself decides that the
marauding hound in question is a supernatural beast, and all he wants
to ask Sherlock Holmes is what to do with the next of kin.
\item 
From Holmes' point of view, every set of clues points toward a
logical, real- world solution. Rejecting the supernatural
explanation, Holmes decides to consider all other options before
falling back on that one. Sherlock Holmes personifies the
intellectual's faith in logic, and on examining facts to find the
answers.
\item 
In this sense, the story takes on the Gothic tradition, a brand of storytelling that highlights the bizarre and unexplained. Doyles' mysterious hound, an ancient family curse, even the ominous Baskerville Hall all set up a Gothic- style mystery that, in the end, will fall victim to Holmes' powerful logic.
\end{itemize}

\newpage
\begin{quotation}
  \eng{The moon was shining bright upon the clearing, and there in the
  centre lay the unhappy maid where she had fallen, dead of fear and
  of fatigue. But it was not the sight of her body, nor yet was it
  that of the body of Hugo Baskerville lying near her, which raised
  the hair upon the heads of these three daredevil roysterers, but it
  was that, standing over Hugo and plucking at his throat, there stood
  a foul thing, a great, black beast, shaped like a hound, yet larger
  than any hound that ever mortal eye has rested upon.}
\end{quotation}
\begin{itemize}
\item a quote from an old manuscript --- distanced from the present
\item In the present they are big, but not impossibly so: 
\\ \eng{Dr
    Mortimer looked strangely at us for an instant, and his voice sank
    almost to a whisper as he answered: 'Mr Holmes, they were the
    footprints of a gigantic hound!'}

\end{itemize}

\myslide{Class and hierarchy}
\begin{itemize}
\item 
Throughout the story, the superstitions of the shapeless mass of
common folk- everyone attributes an unbending faith in the curse to
the commoners-are denigrated and, often, dismissed. If Mortimer and
Sir Henry have their doubts, it is the gullible common folk who take
the curse seriously. 
\item In the end, when Watson's reportage and Holmes'
insight have shed light on the situation, the curse and the commoners
who believed it end up looking silly.
\end{itemize}

\myslide{Hound of the Baskervilles (1939)}
\begin{itemize}
\item the 1939 adaptation \textit{The Hound of the
    Baskervilles} by 20th Century-Fox, starring Basil Rathbone as
  Holmes and Nigel Bruce as Watson.
\item First of 14 films with this pair
\item Set in Victorian England (not remade as contemporary)

\end{itemize}

% \myslide{A Scandal In Bohemia}

% \begin{itemize}
% \item The villain is more sympathetic than the client
% \item Homes is outwitted because he underestimates a woman
%   \begin{itemize}
%   \item this is very atypical

%   \end{itemize}
% \end{itemize}

% \myslide{Why Bohemia?}

% \begin{itemize}
% \item Another story with an ambiguous title
% \item It deals with the King of Bohemia
% \item But we see him living a bohemian lifestyle
%   \begin{quote}
%     Holmes, who loathed every form of society with his whole Bohemian
%     soul, remained in our lodgings in Baker Street,
%   \end{quote}
%   The usage comes from French, where Romani (Gypsies) where though to
%   come to Europe through Bohemia, and lead interesting lives of ill repute.
% \end{itemize}

% \myslide{The Red-Headed League (1891)}

% \begin{itemize}
% \item  The credible narrator credible?

%   \begin{quote}
%     Holmes: “I know, my dear Watson, that you share my love of all
%     that is bizarre and outside the conventions and humdrum routine of
%     everyday life. You have shown your relish for it by the enthusiasm
%     which has prompted you to chronicle, and, if you will excuse my
%     saying so, somewhat to embellish so many of my own little
%     adventures.”
%   \end{quote}
% \item The effect of Holmes’ remark on the reader.
% \end{itemize}

% \myslide{The Red-Headed League}
% (1891)
%  Structural similarities in the stories; Holmes’
% observations, often perceived as genius, often
% downplayed.
%  Very

% much of the construction of the “genius”
% Holmes also attributed to Holmes’ knowledge
% of just about everything.

%  The

% ad that Wilson responded to first
% published in The Morning Chronicle.

%  Calling

% attention to the “chronicle” or
% “chronicling”; that which is factual /
% invented.


% The Red-Headed League
% (1891)
% 

% Depiction of John Clay.

% 

% “I beg that you will not touch me with your filthy hands,'
% remarked our prisoner, as the handcuffs clattered upon his
% wrists. 'You may not be aware that I have royal blood in my
% veins. Have the goodness also when you address me always to
% say “sir” and “please”.”

% 

% Depiction of Peter Jones:

% “This fellow Merryweather is a bank director and personally
% interested in the matter. I thought it as well to have Jones with us
% also. He is not a bad fellow, though an absolute imbecile in his
% profession. He has one positive virtue. He is as brave as a bulldog,
% and as tenacious as a lobster if he gets his claws upon anyone.
% Here we are, and they are waiting for us.”


% The Red-Headed League
% (1891)
% 

% Holmes never has any problems of his own—his life
% boring but made exciting by his clients’ “little
% problems.”

% 

% “l'Homme n'est Rien l'Oeuvre Tout” (the man is
% nothing, the work is everything” misquoted.

% +
% References
% 

% Doyle, Arthur Conan. The Complete Sherlock Holmes.
% Volumes I and II. Introduction and Notes by Kyle
% Freeman. New York: Barnes & Noble Classics, 2003.

% 



% 

% http://www.dailymotion.com/video/x13m7yj_sherlockholmes-12-the-red-headed-league_creation


\myslide{Bibliography}
  \begin{itemize}
  \item   John Scaggs (2005) \textit{Crime Fiction} (CF), Routledge
  \item \textit{The Victorian Web: Literature, History \& Culture in
the Age of Victoria} (VW) Holmes info. contributed by Dr.
Andrzej Diniejko,
\url{http://www.victorianweb.org/authors/doyle/addiction.html}
  \end{itemize}

\end{document}

%%% Local Variables: 
%%% coding: utf-8
%%% mode: latex
%%% TeX-PDF-mode: t
%%% TeX-engine: xetex
%%% End: 

