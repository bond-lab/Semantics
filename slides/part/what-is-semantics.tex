\section{Introduction to Semantics}

\begin{frame}{What is Semantics}
\begin{itemize}
\item Very broadly, semantics is the study of meaning
  \begin{itemize}
  \item Word meaning
  \item Sentence meaning
  \item Contextual meaning (pragmatics)
  \end{itemize}
\item Why do we want to study meaning?
  \begin{itemize}
  \item It underlies our understanding of the world
  \item It is fundamental to our thinking, but we don't consciously know what we are doing
  \end{itemize}
\item What kind of knowledge does it take for a speaker to produce language and for a hearer to comprehend language? 
\end{itemize}

\end{frame}



\begin{frame}{Layers of Linguistic Analysis}
\begin{enumerate}\addtolength{\itemsep}{-0.75ex}
\item Phonetics \& Phonology
\item Morphology
\item Syntax
\item \txx{Semantics}
\item Pragmatics
\item Stylistics
\end{enumerate}
% Two theories
% \begin{itemize}
% \item Semantics is \txx{autonomous}, a separate module
% \item Semantics is \txx{integrated} with other knowledge, inseparable
%   \begin{itemize}
%   \item linguistic knowledge is inseparable from encyclopedic knowledge
%   \end{itemize}
% \end{itemize}
\end{frame}

\begin{frame}[allowframebreaks]{Do people share a common conceptual system?}

\begin{itemize}
\item What is a \lex{high school}?
\item What color is \lex{blue}?
\item What does \lex{verb} mean?
\item What is  \lex{carrot cake}?
\item What color are \lex{traffic lights}?
\end{itemize}

\framebreak

\begin{quotation}
  
Japanese traffic lights are green (as required by international
agreements).  However they are typically called 青い \jpn[blue]{aoi},
the same word as the color of the sky.  Historically this color
historically covered both green and blue ``grue'',
with 緑 \jpn[green]{midori} being a later addition.  For this reason,
the Japanese government decided in 1973 to change the color of the go
light to the bluest possible hue of green!


\hfill \href{https://www.japantimes.co.jp/life/2013/02/25/language/the-japanese-traffic-light-blues-stop-on-red-go-on-what/}{The Japanese traffic light blues: Stop on red, go on what?}
\end{quotation} 
\end{frame}

\begin{frame}{Word Meaning and Sentence Meaning}

\begin{itemize}
\item We store information about words in our \txx{mental lexicon}
  \begin{itemize}
  \item It is still unclear what exactly a word is!
  \end{itemize}
\item Words can be combined to form an infinite number of expressions
  \begin{itemize}
  \item This building up of meaning is referred to as \txx{composition}
  \item If the meaning of the whole can be deduced from the parts then it is \txx{compositional}
  \end{itemize}
\end{itemize}
\end{frame}

\begin{frame}{Reference and Sense}

\begin{itemize}
\item Words \txx{refer} to things in the world (like \iz{unicorn}s)
\item The meaning of a word across different contexts is often referred to as its \txx{sense}
  \begin{itemize}
  \item Same word can refer to different things
    \begin{itemize}
    \item English: \eng{I put my money in the \ul{bank}}
    \item English: \eng{I fell asleep at the river \ul{bank}}
    \end{itemize}
  \item Same basic concept can have different boundaries
    \begin{itemize}
    \item French: \eng[sheep/mutton]{mouton}
    \item English: \eng{sheep} vs \eng{mutton}
      
    \item Japanese: \eng[dove/pigeon]{hato}
    \item English: \eng{dove} vs \eng{pigeon}
    \end{itemize}
  \end{itemize}
\end{itemize}



\end{frame}

\begin{frame}{Representing meaning}
\MyLogo{Also vector space, description, images, video, \ldots}
\begin{itemize}
\item One of our goals will be to represent meaning
\item There are various ways to do this
  \begin{itemize}
  \item Syntactic trees
  \item Logical forms
  \item Thesauri and Ontologies 
  \item Translation
  \item Paraphrasing
  \end{itemize}
Can you think of others?

\item At the end of this course you should be able to use these to
  describe many aspects of word meaning
\end{itemize}


\end{frame}



\begin{frame}{Further Reading}

\begin{itemize}
\item Introduction What does it mean to mean?
  \begin{itemize}
  \item Saeed: \S~1
  \end{itemize}
\end{itemize}
\end{frame}
