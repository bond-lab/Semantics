\section{Word Meaning Revisited}



\begin{frame}{Defining Meaning}

\begin{itemize}
\item When we use a word, we don't have to know everything about the
  referent
  \begin{itemize}
  \item A \eng{dog-cart} is a kind of \con{cart}
  \item[\ent] you can ride it
  \item[\ent] it has wheels
  \end{itemize}
\item We infer that it has many of the same properties as its
  \txx{hypernym}, even though it may not have all
 \begin{itemize}
  \item A \eng{hover-car} is a kind of \con{car}
  \item[\ent] you can ride it
  \item[$\not\Rightarrow$] it has wheels
  \end{itemize}
\item Many of the properties may be irrelevant to the story at hand,
  and irrelevant to the syntax of the language
\end{itemize}

\end{frame}

\begin{frame}{How do we learn?}

\begin{quote}
  \textit{You shall know a word by the company it keeps} \hfill
\citep[p11]{Firth:1957}
\end{quote}

\begin{itemize}
\item You see a new word \emp{in context}
\\ \eng{buttoning up his \ul{pea-jacket}, }
\item And you deduce information from the context

\begin{itemize}
\item[?] it is a kind of jacket \hfill(\eng{yellow jacket}?)
\item[?] with buttons
\item [?] it is thick material (they are going to a stake out)
\item[?] it has something to do with peas \\ $\times$ not true (from the West
  Frisian word \eng{pijjakker}, in which \eng{pij} referred to the
  type of cloth used, a coarse kind of \ul{twilled} blue cloth)
\end{itemize}
\item We are getting better at doing this with computers
\begin{itemize}
\item but people don't just use words
\item they have eyes and noses and other senses, \\ and they have brains
  that link things
\end{itemize}

\end{itemize}

\end{frame}

\begin{frame}{How else do we learn?}

\begin{itemize}
\item From word internal cues
  \begin{itemize}
  \item \eng[far vision]{Television} 
  \item \eng[internet phone]{iphone} 
    (also \eng{individual, instruct, inform, inspire} from the \lex{iMac})
  \item 鯖 \jpn[mackerel]{saba} = 魚 fish; 青 blue
  \end{itemize}
\item From the sound
  \begin{itemize}
  \item \eng{bouba/kiki}  $\star$ or $\clubsuit$
  \item \eng{banged, beaten, battered, bruised, blistered, bashed}
  \item mouth shape for \eng{teeny weeny} vs \eng{large}
  \end{itemize}
\item From  images:
  \begin{tabular}[t]{c}
\includegraphics[width=5em]{pics/magnifying-glass} \\    
 \eng{Magnifying Glass} 
  \end{tabular}
\end{itemize}


\end{frame}

\begin{frame}{Words are related in many other ways}


\begin{itemize}
\item Domains: \eng{ball, racket, net, love, ace}
\item Origin: \eng{chew, eat, drink} vs \eng{masticate, consume, imbibe}
\item[?] come up with some words with different origins\task 
  \\ English or another language!
\item Dialect: \eng{ripper, bonza, sickie, no worries}
\item Part-of-speech: \eng{die, live} vs \eng{death, life}
\item When you learned them!
\item and many more
\end{itemize}

All of these relations affect how you use and understand language.

\end{frame}
