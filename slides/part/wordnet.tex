\section{Wordnet}

\begin{frame}{WordNet}

\MyLogo{\citet{Miller:1998:foreword,_Fellbaum:1998}}

\begin{itemize}
\item WordNet is an open-source electronic lexical database of English,
  developed at Princeton University
  \begin{quote}
    \url{http://wordnet.princeton.edu/}
  \end{quote}
\item Made up of four separate semantic nets, for each of nouns, verbs,
  adjectives and adverbs
\item WordNets exist for many languages, my group has worked on:
  \begin{itemize}
  \item Japanese 
  \item Bahasa Malay/Indonesian
  \item Chinese (Mandarin and Cantonese)
  \item The shared open multi-lingual wordnet (150+ languages) 
    \\ \url{https://omwn.org/}
  \item Kristang
  \item Myanmar
  \item Czech
\end{itemize}
\end{itemize}
\end{frame}

\begin{frame}{Wordnet Structure}
\begin{itemize}
\item Lexical items are categorised into $\sim$115K (and counting) glossed \textbf{synsets} (= synonym sets)

%   \begin{quote}\smaller[2]
% \begin{verbatim}
% 1. enrichment -- (act of making fuller or more 
%    meaningful or rewarding)
% 2. enrichment -- (a gift that significantly increases 
%    the recipient's wealth)
% \end{verbatim}
%   \end{quote}
\item Lexical relations at either the synset level or sense (=
  combination of lexical item and synset) level 
\item Strongly lexicalist (originally):
  \begin{itemize}
  \item synsets only where words exist
  \item but many multiword expressions ($\approx 50\%$)
%  \item (near) absence of frame semantics
  \end{itemize}

% \newpage

% \item Other quirks/properties:
%   \begin{itemize}
%   \item 25 \textbf{unique beginners} in noun semantic net
%   \item taxonomic vs.\ functional hyponymy (cf.\ \eng{chicken} vs.\
%     \eng{bird/food})
%   \item few proper nouns and no separate classification for proper nouns
%   \end{itemize}
\end{itemize}

\end{frame}

\begin{frame}{Psycholinguistic Foundations of WordNet}

\MyLogo{}

\begin{itemize}
\item Strong foundation on hypo/hypernymy (lexical inheritance) based on
  \begin{itemize}
  \item   response times to sentences such as:
    \begin{quote}%\smaller
      \eng{a canary \{can sing/fly,has skin\}}\\
      \eng{a bird \{can sing/fly,has skin\}}\\
      \eng{an animal \{can sing/fly,has skin\}}
    \end{quote}
  \item analysis of anaphora:
    \begin{quote}%\smaller
      \eng{I gave Kim a novel but the \{book,?product,...\}
        bored her}\\
      \eng{Kim got a new car. It has shiny \{wheels,?wheel nuts,...\}}
    \end{quote}
  \item selectional restrictions
  \end{itemize}
\item Is now often used to calculate \txx{semantic similarity}
  \begin{itemize}
  \item The shorter the path between two synsets the more similar they are
  \item Or the shorter the path to the nearest shared hypernym, \ldots
  \end{itemize}
\end{itemize}
\end{frame}

\begin{frame}{Word Meaning as a Graph}
\MyLogo{It ends up being a very big graph}
%\includegraphics[height=0.75\textheight]{img/Semantic_Net.eps}
 \begin{forest}
   [entity
   [object [artifact [equipment [golf-club [driver,name=d1] ] ] ]
           [agent [person [worker [driver,name=d2] ]]] ]
           [process
       [move [propel [swing  [drive,name=d11] ]]]
       [manipulate [  control [operate [drive,name=d21] ] ] ] ]
        [substance]
       [abstract [communication [writing [code [driver ] ] ] ]]
    ] 
 \draw[dashed] (d1) to  [out=-45, in=-135]  node[yshift=-1ex]{instrument} (d11);
 \draw[double] (d2) to  [out=-45, in=-135]  node[yshift=-1ex]{agent} (d21);
 ]
 \end{forest}
 % [object [artifact [equipment [golf-club [driver,name=d1] ] ] ]
      %  agent [person [worker [driver,name=d2] ]]]
      %  [process
      %  [move [propel [swing  [drive,name=d11] ]]]
      %  [manipulate [  control [operate [drive,name=d21] ] ] ] ]
       
      %  [substance]
      %  [abstract [communication [writing [code [driver ] ] ] ]]
    %] 
 % \draw[dashed] (d1) to  [out=0, in=-180] (d11);
 % \draw[double] (d2) to [out=-75, in=110](d21);
%\end{forest} 

\end{frame}

% \begin{frame}{Word Meaning as a Graph}
% \MyLogo{}
% \bigskip
% \bigskip
% \bigskip
% \includegraphics[height=0.75\textheight]{img/Semantic_Net.eps}

% \begin{itemize}
% \item You need a very big graph to capture all meanings
% \end{itemize}

%\end{frame}


\begin{frame}{Wordnet in this course}
%%% FIXME inheritance

\begin{itemize}
\item We will use wordnet to test our skills in determining word meaning
  \begin{itemize}
  \item tag a short text from this year's story or stories
  \item discuss differences with other annotators
 \end{itemize}
\item As well as a source of examples and inspiration
\item  my students have used wordnets for:
  \begin{itemize}
  \item  Japanese derivational
relations \citep{Bond:Wei:2019}
\item  pronoun representation for Japanese, Mandarin and English
  \citep{Seah:Bond:2014}
\item exclamatives and classifiers
\citep{Mok:Gao:Bond:2012,daCosta:Bond:2016}
\item  sentiment analysis
  \citep{Le:Moeljadi:Miura:Ohkuma:2016,Bond:Janz:Piasecki:2019}
\item cross-lingual sense annotation \citep{Bonansinga:Bond:2016}
\item multilingual crosswords \citep{Tan:2012}
\item [\ldots]
\end{itemize}
\end{itemize}
\end{frame}
