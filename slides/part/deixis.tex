\section{Deixis}

\begin{frame}{What is Deixis}
\begin{itemize}
\item any linguistic element whose interpretation
  necessarily makes reference to properties of the
  extra-linguistic context in which it occurs is \txx{deictic}
  \begin{description}
  \item[\txx{Person}] relative to the speaker and addressee; \eng{you, me, them}
  \item[\txx{Spatial Location}] demonstratives; \eng{this, that, over there, here}
  \item[\txx{Temporal Location}] tense; \eng{yesterday, today, tomorrow}
  \item[\txx{Social Status}] relative to the social position: \eng{professor, you, uncle, boy}
  \end{description}
\item \txx{Discourse deixis}: referring to a linguistic expression or chunk of discourse
\end{itemize}

\newpage
More than 90\% of the declarative sentences people utter are indexical
in that they involve implicit references to the speaker, addressee,
time and/or place of utterance in expressions like first and second
person pronouns, demonstratives, tenses, and adverbs like \lex{here}, \lex{now},
\lex{yesterday} \citep[p366]{Bar-Hillel:1954}.

\end{frame}

\begin{frame}{Spatial Deixis}
\begin{itemize}
\item Two way systems (English, \ldots)
  \\[2ex] \begin{tabular}{llll}
    \txx{proximal} &\lex{this} & \lex{here} &close to the speaker\\
    \txx{distal} &\lex{that} & \lex{there} & far from the speaker 
  \end{tabular}
\item Three (four) way systems (Japanese, \ldots)
    \\[2ex] \begin{tabular}{lllll}
                    & Gloss & \con{thing}   & \con{place}  \\
\hline
      \txx{proximal}  & close to speaker & \lex{kore} ``this'' & \lex{koko} ``here''\\
      \txx{medial} &close to addressee &\lex{sore} ``that''   & \lex{soko} ``there'' \\
      \txx{distal} &far from both&\lex{are} ``\,'tother'' & \lex{asoko} ``over there''  \\ \hline
      \txx{Q} & interrogative & \lex{dore} ``what'' & \lex{doko} ``where''
  \end{tabular}
\end{itemize}
\begin{taskb}
   Can you do English \con{time}? 
\end{taskb}
\begin{taskb}
   Can you do these in another language? 
\end{taskb}
 % Can decompose: \lex{here} ``this place'', \lex{there} ``that place'', \lex{where} ``what place''
 % \lex{now} ``this time'',  \lex{then} ``that time'', \lex{when} ``what time''

 % \noindent\begin{tabular}{lllll}
%     & close to sp  & \multicolumn{2}{c}{far from sp} & Q \\  
%     & & close to hr & far from both \\ \hline
%   English & this & \multicolumn{2}{c}{that} & what \\
%   \ \ \ \  place & here & \multicolumn{2}{c}{there} & where \\
%   Japanese & kono & sono & ano & dono \\
%    \ \ \ \  place & koko & soko & asoko & doko \\
% \end{tabular}


% \mytask{try this}

% \begin{tabular}[l]{}
%   Q & close & far & further \\
%   what & this & that & 'tother \\
%   ?    & now  & then & --- \\
%   where & ? & there & over there \\
% \end{tabular}


\end{frame}

\begin{frame}{More Spatial Deixis}

\begin{itemize}
\item Often lexicalized:
  \begin{itemize}
  \item \lex{go, come, foreign, home, local, indigenous, national language}
  \end{itemize}
\item Can lead to \txx{discourse}/\txx{textual deixis}
  \begin{exe}
    \ex \eng{\ul{Here} we begin explaining textual deixis}
  \end{exe}
\item Often also used for time
  \begin{exe}
    \ex \eng{\ul{This year} we are trying a new kind of assignment}
  \end{exe}
\newpage
\item Spatial expressions extend to possession in many languages
  \begin{exe}
    \ex \gll \jpn{NICT-ga} \jpn{Kyoto-ni} \jpn{aru} \\
    NICT-\textsc{nom} Kyoto-\textsc{loc} be \\
    \trans NICT is in Kyoto
    \ex \gll \jpn{watashi-ni} \jpn{musuko-ga}  \jpn{aru} \\
     I-\textsc{loc}  son-\textsc{nom} be \\
    \trans I have a son (lit. a son is in me)
   \end{exe}
\end{itemize}

\end{frame}

\begin{frame}{Person Deixis}

\begin{itemize}
\item Minimally a three way division
\\[2ex]  \begin{tabular}{lll}
    First Person & Speaker & \lex{I} \\
    Second Person & Addressee & \lex{you} \\
    Third Person & Other & \lex{he/she/it} \\
  \end{tabular}
\item Often combined with
  \begin{itemize}
  \item \txx{gender}: \lex{he/she/it }
  \item \txx{number}: \lex{I/we}, 
    \lex{'anta} ``you:m'', \lex{'antumaa} ``you:dual'',  \lex{'antum} ``you:m:pl''
    \\ (Arabic)
  \item \txx{inclusion}: \lex{n\'uy} ``we including you'',  \lex{n\'{\i}i} ``we excluding you'' (Zayse)
  \item \txx{honorification}: \lex{kimi} ``you:inferior'', \lex{anata} ``you:equal'',
    \\ don't use pronouns for superiors: \lex{sensei} ``teacher'', \ldots (Japanese)
  \end{itemize}
\end{itemize}

\end{frame}

\begin{frame}{Social Deixis}

In European languages, a two-way choice in 2nd person pronominal
reference: the T/V distinction

\begin{itemize}
\item T/V distinctions in European languages
  \\[2ex]
  \begin{tabular}{lll}
    & Familiar 2sg & Polite 2sg \\ \hline
    French & \lex{tu} & \lex{vous} \\
    German & \lex{du} & \lex{Sie} \\
%    Spanish & \lex{t\'u} & \lex{usted} \\
    Czech & \lex{ty} & \lex{vy} \\ 
  \end{tabular}

\item Shift from asymmetric use showing \txx{power} (superior uses \lex{tu}; inferior uses \lex{vous}) to symmetric use showing \txx{solidarity} (strangers use  \lex{vous}; intimates use \lex{tu}): typically the socially superior person must invite the socially
  inferior person to use the familiar form
\end{itemize}

\end{frame}

\begin{frame}{Social Deixis can be marked on other words}
\MyLogo{It must be marked}
\begin{exe}
  \ex \jpn{Tanaka-san-ga kudasaimashita} \hfill [addressee and subject hon.]
  \trans Tanaka gave it to me (and I honor him and you)
%  \trans
 \ex \jpn{Tanaka-san-ga kudasatta} \hfill [subject honorification]
  \trans Tanaka gave it to me (and I honor him)

 \ex \jpn{Tanaka-kun-ga kuremashita} \hfill [addressee honorification]
  \trans Tanaka gave it to me (and I honor you)

 \ex \jpn{Tanaka-kun-ga kureta} \hfill [no honorification]
  \trans Tanaka gave it to me (implies I am higher status than him)

\end{exe}

\begin{taskb}
% \item Find examples where someone addresses Sherlock as \eng{Holmes}
%   and compare then to examples where he is addressed as \eng{Mr
%     Holmes}: what is the difference? \task
Find examples in \textit{Válka s Mloky} or another text where \cs{ty} and \cs{vy} are used:  what is the difference? 
\end{taskb}

\end{frame}

\begin{frame}{Further Reading}
\begin{itemize}
\item Deixis
  \begin{itemize}
  \item Saeed: \S~7.2
  \end{itemize}
\end{itemize}


\end{frame}
