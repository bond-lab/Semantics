
\begin{frame}{Derivational Relations}

\begin{itemize}
\item Often words are linked by more or less systematic relations, sometimes morphologically marked
  \begin{itemize}
  \item \lex{beauty}/\lex{beautiful}/\lex{beautify}
  \item \lex{cute}/\lex{cuteness}
  \end{itemize}
\end{itemize}
\end{frame}

\begin{frame}{Agentive Nouns}
\MyLogo{}
\begin{itemize}
\item An \txx{agentive noun} is a word that is typically derived from
  another word denoting an action, and that identifies an entity that
  does that action.  \\ \textbf{verb} + \lex{-er, -or, -ant}
  \begin{exe}
    \ex \lex{murderer, commentator, whaler, director, computer}
    \ex ?? \lex{undertaker, cooker, footballer} (Saeed also includes these)
  \end{exe}
\item Should \lex{murderer} be listed separately from \lex{murder} in
the dictionary? Why or why not?
\item Also \txx{recipient noun}s that show the undergoer:
  \textbf{verb} + \lex{-ee}: \lex{employee}, \lex{trustee}
\end{itemize}
\end{frame}

\begin{frame}{Agentive Nouns in Other Languages}
\MyLogo{Thanks to Yeo Jia Qi (Malay)}

\begin{itemize}
\item Japanese (suffix distinguishes person/machine)
  
  \begin{itemize}
  \item \jpf{運転する} → \jpf{運転者} \eng[driver]{unten-sha}
  \item \jpf{計算する} → \jpf{計算者} \jpf{計算機} \eng[computer]{keisan-sha/ki}
  \item \jpf{研究する} → \jpf{研究者} \jpf{研究員} \eng[researcher]{kenkyuu-sha/in}
  \item \jpf{読む} → \jpf{読み手} \jpf{読者} \jpn[reader]{yomite/dokusha}
  \end{itemize}
\item Malay (prefix can convert any part  of speech)
  \begin{itemize}
  \item \eng[help]{bantu (v)} → \eng[assistant/helper]{pembantu} 
  \item \eng[cut]{potong (v)} → \eng[cutter (human/machine)]{pemotong}
  \item \eng[fly]{terbang (v)} → \eng[pilot (not passenger)]{penerbang}
%  \item \zsm[scissors]{gunting (n)} → \zsm[(editor – human)]{penyunting}
  \end{itemize}
\end{itemize}
\end{frame}

