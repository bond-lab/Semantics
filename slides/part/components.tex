\section{Componential Analysis}


\begin{frame}{Break word meaning into its components}
\MyLogo{Inspired by work on phonetics in the Prague School}
\begin{itemize}
\item For example:
  \\[2ex] \begin{tabular}{lllll}
    \lex{woman} & \cmp{female} & \cmp{adult} & \cmp{human} & \\
    \lex{spinster} & \cmp{female} & \cmp{adult} & \cmp{human} & \cmp{unmarried} \\
    \lex{man} & \cmp{male} & \cmp{adult} & \cmp{human} & \\
    \lex{bachelor} & \cmp{male} & \cmp{adult} & \cmp{human} & \cmp{unmarried} \\
    \lex{wife} & \cmp{female} & \cmp{adult} & \cmp{human} & \cmp{married} \\
    \lex{girl} & \cmp{female} & \cmp{child} & \cmp{human} & \\
    \lex{boy} & \cmp{male} & \cmp{child} & \cmp{human} & \\
  \end{tabular}
  \\[2ex] \txx{semantic components}/\txx{primitives} shown as \cmp{component}
  \begin{itemize}
  \item components allow a compact description
  \item interact with morphology/syntax
  \item form part of our cognitive architecture
  \end{itemize}
\end{itemize}
\end{frame}

\begin{frame}{Defining Relations using Components}

\begin{itemize}
\item \txx{hyponymy}
  \begin{quote}
    A lexical item P is a hyponym of Q if all the components of Q are also in P.
  \end{quote}
  \begin{tabular}{lllll}
    \lex{woman} & \cmp{female} & \cmp{adult} & \cmp{human} & \\
    \lex{spinster} & \cmp{female} & \cmp{adult} & \cmp{human} & \cmp{unmarried} \\
%    \lex{bachelor} & \cmp{male} & \cmp{adult} & \cmp{human} & \cmp{unmarried} \\
    \lex{wife} & \cmp{female} & \cmp{adult} & \cmp{human} & \cmp{married} \\
  \end{tabular}
\\[2ex]  \lex{spinster} $\subset$ \lex{woman}; \lex{wife} $\subset$ \lex{woman}
\item \txx{incompatibility}
  \begin{quote}
    A lexical item P is incompatible with Q if they share some
    components but differ in one or more \txx{contrasting} components
  \end{quote}
%   \begin{tabular}{lllll}
% %    \lex{woman} & \cmp{female} & \cmp{adult} & \cmp{human} & \\
%     \lex{spinster} & \cmp{female} & \cmp{adult} & \cmp{human} & \cmp{unmarried} \\
% %    \lex{bachelor} & \cmp{male} & \cmp{adult} & \cmp{human} & \cmp{unmarried} \\
%     \lex{wife} & \cmp{female} & \cmp{adult} & \cmp{human} & \cmp{married} \\
%   \end{tabular}
  \lex{spinster} $\not\approx$ \lex{wife}

\end{itemize}
\end{frame}

\begin{frame}{Binary Features}

\begin{itemize}
\item We can make things more economical (fewer components):
  \\[2ex] \begin{tabular}{lllll}
    \lex{woman} & \cmp{+female} & \cmp{+adult} & \cmp{+human} & \\
    \lex{spinster} & \cmp{+female} & \cmp{+adult} & \cmp{+human} & \cmp{--married} \\
    \lex{bachelor} & \cmp{--female} & \cmp{+adult} & \cmp{+human} & \cmp{--married} \\
    \lex{wife} & \cmp{+female} & \cmp{+adult} & \cmp{+human} & \cmp{+married} \\
    \lex{girl} & \cmp{+female} & \cmp{-adult} & \cmp{+human} & \\
  \end{tabular}
  \begin{itemize}
  \item Which should be $+$? \cmp{+female} or \cmp{--male}
  \item Presumably also \cmp{--electric}, \cmp{--conical}, \ldots
    \\ Only show \txx{relevant} features
  \item \txx{antonyms} differ in only one binary component
  \end{itemize}
\end{itemize}
\end{frame}

\begin{frame}{Redundancy Rules}

\begin{itemize}
\item We can add relations between components:
\\[2ex]  \begin{tabular}{llll}
     \cmp{+human} & \into & \cmp{+animate}  \\
     \cmp{+adult} & \into & \cmp{+animate}  \\
     \cmp{+animate} & \into & \cmp{+concrete}  \\
     \cmp{+married} & \into & \cmp{+adult}  \\
     \cmp{+married} & \into & \cmp{+human}   & \ldots
  \end{tabular}
\item Which allows us to write:
  \\[2ex] \begin{tabular}{lllll}
    \lex{woman} & \cmp{+female} & \cmp{+adult} & \cmp{+human} & \\
    \lex{spinster} & \cmp{+female} & \cmp{+adult} & \cmp{+human} & \cmp{--married} \\
    \lex{bachelor} & \cmp{--female} & \cmp{+adult} & \cmp{+human} & \cmp{--married} \\
    \lex{wife} & \cmp{+female} & &   & \cmp{+married} 
  \end{tabular}
  \\[2ex] Can we say  \cmp{--married}  \into\  \cmp{+human}?
\end{itemize} 
\end{frame}

\begin{frame}{More Complex Breakdowns}

\begin{itemize}
\item We can add relations between components:
\\[2ex]  \begin{tabular}{llll}
     \cmp{+father} & \into & \cmp{+male} \cmp{+parent}  \\
     \cmp{+father}(\ul{x},y) & \into & \cmp{+male}(x) \cmp{+parent}(x,y) \\
     \cmp{+son}(\ul{x},y) & \into & \cmp{+male}(x) \cmp{+parent}(y,x) \\
     \cmp{+brother}(\ul{x},y) & \into & \cmp{+male}(x) 
     \cmp{+parent}(z,x) \cmp{+parent}(z,y) \\
     \cmp{+grandfather}(\ul{x},y) & \into & \cmp{+male}(x) 
     \cmp{+parent}(x,z)  \cmp{+parent}(z,y) \\
  \end{tabular}
\item Assume \cmp{+parent}(x,y)  means ``x is the parent of y''
\item There are various ways you can formalize such relationships
  \begin{itemize}
  \item Many parts of language can be formalized in such a way
  % \item We did this for demonstratives
  %   \\ \lex{this, that, these, those, what, here, there, where}\task
  \end{itemize}
\end{itemize}


These are great for many sub-systems of language, but it is hard to make components for everything, \ldots


\end{frame}

\begin{frame}{Word Embeddings}
\MyLogo{Slides based on \href{https://www.shanelynn.ie/get-busy-with-word-embeddings-introduction/}{An introduction to word embeddings for text analysis} by Shane Lynn (2017)}


\begin{itemize}
\item Represent words as a vector of numbers (instead of a set of components)
\item Every word has a unique word embedding (or “vector”), which is
  just a list of numbers for each word.  
\item Embeddings start being useful from 50-500 dimensions
  \\ LLMs typically are much larger
\item The embedding captures the  “meaning” of the word. 
\item  Similar words end up with similar embedding values
\item Context based word embeddings give a different vector depending
  on the context 

\end{itemize}

More later, \ldots
\end{frame}
