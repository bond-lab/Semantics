
\section{The meanings of words}


\begin{frame}[allowframebreaks]{Words carry different meanings: \lex{leave}}
\MyLogo{\url{https://compling.upol.cz/ntumc/cgi-bin/showcorpus.cgi?sid_from=10000&sid_to=11000&clemma=leave}}
\begin{description} %\addtolength{\itemsep}{0ex}
\item[10070] \eng{Nothing was \ul{left} save a few acres of ground , and the
  two-hundred-year-old house , which is itself crushed under a heavy
  mortgage .}
% leave (have left or have as a remainder) 
\item[10079] \eng{The money which my mother had \ul{left} was enough
  for all our wants , and there seemed to be no obstacle to our
  happiness . "}
%leave (leave or give by will after one's death;) 
\item[10085] \eng{He had no friends at all save the wandering
  gipsies , and he would give these vagabonds \ul{leave} to encamp upon
  the few acres of bramble- covered land which represent the family
  estate , and would accept in return the hospitality of their tents
  , wandering away with them sometimes for weeks on end .}
% leave (permission to do something;) 
\item[10107] \eng{She
  \ul{left} her room , therefore , and came into mine , where she sat for
  some time , chatting about her approaching wedding .}
%leave (move out of or depart from;) 
\item [10108]\eng{At
  eleven o'clock she rose to \ul{leave} me , but she paused at the door
  and looked back.}
%leave (move out of or depart from;) 
\item[10439] \eng{" The rest you will \ul{leave} in our hands
  . "}
% leave (put into the care or protection of someone;) 
\item[10449] \eng{And now , Miss Stoner , we must \ul{leave} you for if
  Dr. Roylott returned and saw us our journey would be in vain .
  }
% leave (move out of or depart from;) 
\item[10526] \eng{Then he turned down the lamp , and we were \ul{left} in darkness
  .}
% leave (act or be so as to become in a specified state;) 

\end{description}


How many different meanings?\task

\bigskip
From the \href{https://compling.upol.cz/ntumc/cgi-bin/showcorpus.cgi?sid_from=10000&sid_to=11000&clemma=leave}{NTU Multilingual Corpus} (\textit{Adventure of the Speckled band}, concept lemma = \eng{leave})
\end{frame}

\begin{frame}{How can we represent the differences?}

\begin{itemize}
\item Definitions
\item Translations/paraphrases
\item Semantic Relations
\item Components
\item Word Embeddings
\end{itemize}

\end{frame}

\begin{frame}{Semantic Representations of Words}

\begin{itemize}\addtolength{\itemsep}{-1ex}
\item Divide meaning into
  \begin{itemize}
  \item \txx{reference}: the relation to the world/mental space
  \item \txx{sense}: the rest of the meaning
    \begin{itemize}
    \item \txx{denotation} the part that distinguishes the meaning
      from other meanings
    \item \txx{connotation}  cultural or emotional associations 
    \end{itemize}
  \end{itemize}
\item Introduce \con{concepts} %\hfill (meaning as font-change)
  \begin{itemize}
  \item How can we represent concepts?
  \item How do we learn them?
    \begin{itemize}
    \item Typically children start off by \txx{underextending} or \txx{overextending} concepts
    \end{itemize}
  \end{itemize}
\item Example: \eng{That dog}
  \begin{itemize}
  \item reference --- the animal over there
  \item sense --- canine quadruped domesticated by man
  \item connotation --- faithful, friendly (or dirty)
  \end{itemize}
\end{itemize}


\end{frame}

\begin{frame}{Definitional Semantics}

%\MyLogo{\citet{Jackson:2002,Wilks+:1996}}

  \begin{itemize}
  \item Standard lexicographic approach to lexical semantics:
    \begin{quote}
      \textbf{semantics} = \eng{the \textcolor{blue}{study} of \textcolor{brown}{language meaning}}\\
      \textbf{tailor} = \eng{a \textcolor{blue}{person} whose \textcolor{brown}{occupation is making and altering garments}}
    \end{quote}
  \item Definitions are conventionally made up of;
    \begin{itemize}
    \item \textcolor{blue}{genus}: what class the lexical item belongs to
    \item \textcolor{brown}{differentiae}: what attributes distinguish it from
      other members of that class
    \end{itemize}
    % \item ``Decoding'' vs.\ ``encoding'' dictionaries
  \item Often hard  to understand if you don't already know the meaning!
  \end{itemize}
\end{frame}

\begin{frame}{Definitional Semantics: pros and cons}
  \begin{itemize}
  \item Pros:
  \begin{itemize}
  \item familiarity (we are taught to use dictionaries)
  \end{itemize}
\item Cons:
  \begin{itemize}
  \item subjectivity in sense granularity (splitters vs.\ lumpers) and
    definition specificity
  \item circularity in definitions
  \item consistency, reproducibility, \ldots
  \item often focus on diachronic (historical) rather than synchronic (current) semantics 
  \end{itemize}
\end{itemize}
\end{frame}

\begin{frame}[allowframebreaks]{Entries for \lex{leave}}
%\MyLogo{\url{https://compling.upol.cz/ntumc/cgi-bin/cgi-bin/wn-gridx.cgi?gridmode=ntumc-noedit&lang=eng&lemma=leave}}
\begin{description}
%  00613683-v (63)
% V2 	leave 	     	go and leave behind, either intentionally or by neglect or forgetfulness
\item [02015598-v] (72)
V1, V2 	\lex{get out, go out, leave, exit} ``move out of or depart from''

%01494310-v (328)
% V2 	put, place, set, lay, position, pose, leave 	     	put into a certain place or abstract location
\item [02356230-v] (8)
V3 	\lex{leave, entrust} ``put into the care or protection of someone''
% 00053097-n (2)
% 	parting, farewell, leave, leave-taking 	     	the act of departing politely
      \item [02009433-v] (149)
V1 	\lex{leave, go away, go forth} ``go away from a place''
% 15139130-n (3)
% 	leave, leave of absence 	     	the period of time during which you are absent from work or duty
\item[02229055-v] (7)
V3 	\lex{leave, will, bequeath} ``leave or give by will after one's death''
% 02383440-v (14)
% V1, V2 	leave, depart, pull up stakes 	     	remove oneself from an association with or participation in
% 00360092-v (5)
% V2 	leave, leave behind 	     	be survived by after one's death
% 02721438-v (30)
% V2 	leave, allow for, allow, provide 	     	make a possibility or provide opportunity for; permit to be attainable or cause to remain
% 00136991-v (27)
% V2 	leave, leave alone, leave behind 	     	leave unchanged or undisturbed or refrain from taking
\item [02729414-v] (56) V2 	\lex{leave} ``act or be so as to become in a specified state''
\item [02730135-v] (5) V2 \lex{leave} ``have left or have as a remainder''

% 02635659-v (89)
% V2 	result, lead, leave 	     	have as a result or residue
% 02296153-v (27)
% V2, V3 	give, impart, leave, pass on 	     	transmit (knowledge or skills)
% 00613018-v (3)
% V2
\item [06690114-n] (1) \lex{leave} ``permission to do something''
\end{description}

Not to be confused with \eng{\ul{left} hand} and \eng{the \ul{leaves} fell}, \ldots .
\end{frame}

\begin{frame}{Paraphrases and translation}

\begin{itemize}
\item Saying the same thing in different words
  \begin{itemize}
  \item Same language = \txx{paraphrase}
  \item Different language = \txx{translation}
  \end{itemize}
\item We showed some paraphrases in the entries given above
\item If you speak another language, then you can use that to
  disambiguate may things.
  \begin{itemize}
  \item \lex{leave, entrust} = 預ける \jpn{azukeru}
  \item \lex{get out, go out, leave, exit} = 去る \jpn{saru}
  \item \lex{leave, will, bequeath} = 遺す \jpn{nokosu}
  \end{itemize}
  \begin{taskb}
    Can you explain the ambiguity in \eng{The money which my mother
    had \ul{left} was enough for all our wants}?
  \end{taskb}
\end{itemize}
\end{frame}

% \begin{frame}{Paraphrase cues}

% \begin{itemize}
% \item \lex{that is to say}
%   \begin{itemize}
%   \item  \eng{I met him that night, and he called next day to ask if we had got home all safe, and after that we met him - that is to say, Mr Holmes, I met him twice for walks, \ldots} IDEN
%   \item \eng{In three days, that is to say on Monday next  \ldots} FINA
%   \end{itemize}
% \item \lex{in other words}
%   \begin{itemize}
%   \item \eng{\ldots there is a possibility that these initials are those of the second person who was present - in other words, of the murderer. \ldots} BLAC
%   \item \eng{\ldots when they closed their League offices that was a sign that they cared no longer about Mr. Jabez Wilson's presence; in other words, that they had completed their tunnel.} REDH
%   \end{itemize}
% \end{itemize}

% \end{frame}
