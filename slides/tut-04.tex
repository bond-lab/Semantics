\documentclass[a4paper]{article}

\title{Semantics: Tutorial Four}
\author{Francis Bond \url{<bond@ieee.org>}}
\date{}%2011-08-15}
\usepackage{fontenc}
\usepackage{polyglossia}
\setmainlanguage{english}
\setmainfont{TeX Gyre Pagella}
\setsansfont[Ligatures=TeX]{TeX Gyre Heros}
\usepackage{xeCJK}
\setCJKmainfont{Noto Sans CJK JP}
\setCJKsansfont{Noto Sans CJK JP}
\setCJKmonofont{Noto Sans CJK JP}
%\newcommand{\ans}[1]{\hfill{#1}}
%\newcommand{\ans}[1]{}
%%% dynamically add answers
\input{answers}

\usepackage{xspace}
\newcommand{\ent}{\ensuremath{\Rightarrow}\xspace}
\newcommand{\nent}{\ensuremath{\not\Rightarrow}\xspace}
\newcommand{\Y}[1]{\textbf{#1}}

\usepackage{multicol}
%\Restriction{}
%\rightfooter{}
%\leftheader{}
%\rightheader{}
\usepackage{mygb4e}
\newcommand{\lex}[1]{\textbf{\textit{#1}}}
\newcommand{\lx}[1]{\textbf{\textit{#1}}}
\newcommand{\ix}{\ex\it}
\newcommand{\eng}{\textit}
\newcommand{\con}[1]{\textsc{#1}}
\usepackage{url}
\usepackage[normalem]{ulem}
\newcommand{\ul}[1]{\uline{#1}}
\newcommand{\txx}[1]{\textbf{#1}}

\begin{document}
\maketitle

\begin{enumerate}
  
\item Find at least one example each of words with positive and
  negative sentiment from your story.
  \begin{itemize}
  \item Do this ahead of class
  \item Note the entire sentence
  \end{itemize}
   \emph{In the morning the captain returned; he was \ul{angry} and \ul{unhappy}, but said nothing.} (negative, negative)
  \\ \emph{Fred is my \ul{best friend} after all.} (positive)


\item \label{q:valid-argument} Which of the following arguments is \textbf{valid}?
\begin{enumerate}
\item If it rains, the ground gets wet. It is raining. Therefore, the ground gets wet. \label{q:valid-argument-a}
\item If it rains, the ground gets wet. The ground is wet. Therefore, it must have rained.
\item If it rains, the ground gets wet. It is sunny. Therefore, the ground is wet.
\end{enumerate}
\abox{Answer: (\ref{q:valid-argument-a})}

\item \label{q:true-statement} Which statement must be \textbf{true}?
\begin{enumerate}
\item All dogs are mammals. \label{q:true-statement-a}
\item Some dogs are reptiles.
\item All mammals are dogs.
\end{enumerate}
\abox{Answer: (\ref{q:true-statement-a})}

\item \label{q:inclusive-or} In logic, "\textbf{or}" (inclusive) means:
\begin{enumerate}
\item One or the other, but not both.
\item Either one or both. \label{q:inclusive-or-b}
\item Neither one nor the other.
\end{enumerate}
\abox{Answer: (\ref{q:inclusive-or-b})}

\newpage
  
\item Are the following quantifiers:
  \begin{itemize}
  \item[(i)] \textbf{Symmetrical or asymmetrical}?  
  (Symmetrical means switching subject and predicate does not change truth; asymmetrical means it does.)
\item[(ii)] How do they behave with entailments?
  \begin{itemize}
  \item[↑] Upward entailment means true for a more general thing (a hypernym)
  \item[↓] Downward entailment means true for a more specific thing (a hyponym))
  \end{itemize}
    \begin{itemize}
    \item[(iib)] In the \textbf{left argument} (the restrictor): is the quantifier upward or downward entailing?  
    \item[(iic)] In the \textbf{right argument} (the scope): is the quantifier upward or downward entailing?  
    \end{itemize}
  \end{itemize}
\begin{exe}
\ex \textit{most}
     \abox{\begin{itemize}
       \item Asymmetrical (e.g., Most cats are pets ≠ Most pets are cats).
       \item L↑   Most black cats are pets \nent  Most cats are pets 
       \item L↓   Most cats are pets \nent  Most black cats are pets
       \item \Y{R↑}   Most cats are dangerous pets \ent  Most cats are pets
       \item R↓   Most cats are pets \nent  Most cats are  dangerous pets
     \end{itemize}}
\ex \textit{a few (cardinal) ``two or threee''}
     \abox{\begin{itemize}
       \item Symmetrical (e.g., A few cats are pets =  A few pets are cats).
         \\     although we would probably say 2 or 3 not \eng{a few}
       \item \Y{L↑}   A few black cats are pets \ent  A few cats are pets
         \\ although there could be more
       \item L↓   A few  cats are pets \nent  A few black cats are pets
       \item \Y{R↑}   A few cats are dangerous pets \ent  A few cats are pets
                  \\ although there could be more
       \item R↓   A few cats are pets \nent  A few cats are  dangerous pets
     \end{itemize}}
\ex \textit{few (proportional) ``a relatively small amount''}
\abox{\begin{itemize}
      \item Symmetrical (e.g., Few cats are pets ≠ Few pets are cats).
       \item L↑   Few black cats are pets \nent  Few cats are pets
       \item \Y{L↓}   Few cats are pets \nent  Few black cats are pets
       \item R↑   Few cats are dangerous pets \nent  Few cats are pets
       \item \Y{R↓}   Few cats are pets \ent  Few cats are  dangerous pets
     \end{itemize}}
\ex \textit{every}
     \abox{\begin{itemize}
       \item Asymmetrical (e.g., Every cat is an animal ≠ Every animal is a cat).
       \item L↑   Every black cat is a pet \nent  Every cat is a  pet
       \item \Y{L↓}   Every cat is a pet \ent  Every black cat is a  pet
       \item \Y{R↑}   Every cat is a dangerous pet \ent  Every cat is a  pet
       \item R↓   Every cat is a pet \nent  Every  cat is a   dangerous pet
     \end{itemize}}
\ex \textit{at least two}
\abox{\begin{itemize}
       \item Symmetrical (e.g., At least two cats are pets ≈ At least two pets are cats).
       \item \Y{L↑}   At least two black cats are pets \ent  At least two cats are pets
       \item L↓   At least two  cats are pets \nent  At least two black cats are pets
       \item \Y{R↑}   At least two cats are dangerous pets \ent  At least two cats are pets
       \item R↓   At least two cats are pets \nent  At least two cats are  dangerous pets
     \end{itemize}}
   \ex \textit{exactly two}
   \abox{\begin{itemize}
     \item Symmetrical (e.g., Exactly two cats are pets ≈ Exactly two pets are cats).
       \\ normal assumption is that you pick out the same intersective subsets
     \item L↑   Exactly two black cats are pets \nent  Exactly two cats are pets
     \item L↓   Exactly two  cats are pets \nent  Exactly two black cats are pets
     \item R↑   Exactly two cats are dangerous pets \nent  Exactly two cats are pets
     \item R↓   Exactly two cats are pets \nent  Exactly two cats are  dangerous pets
     \item \eng{exactly is a very different kind of quantifier}
     \end{itemize}}
 \end{exe}



 
\end{enumerate}
\end{document}

%%% Local Variables: 
%%% coding: utf-8
%%% mode: latex
%%% TeX-PDF-mode: t
%%% TeX-engine: xetex
%%% End: 
