\documentclass[a4paper]{article}

\title{Semantics: Tutorial Four}
\author{Francis Bond \url{<bond@ieee.org>}}
\date{}%2011-08-15}
\usepackage{fontenc}
\usepackage{polyglossia}
\setmainlanguage{english}
\setmainfont{TeX Gyre Pagella}
\setsansfont[Ligatures=TeX]{TeX Gyre Heros}
\usepackage{xeCJK}
\setCJKmainfont{Noto Sans CJK JP}
\setCJKsansfont{Noto Sans CJK JP}
\setCJKmonofont{Noto Sans CJK JP}
%\newcommand{\ans}[1]{\hfill{#1}}
%\newcommand{\ans}[1]{}
%%% dynamically add answers

\input{answers}

\usepackage{multicol}
%\Restriction{}
%\rightfooter{}
%\leftheader{}
%\rightheader{}
\usepackage{mygb4e}
\newcommand{\lex}[1]{\textbf{\textit{#1}}}
\newcommand{\lx}[1]{\textbf{\textit{#1}}}
\newcommand{\ix}{\ex\it}
\newcommand{\con}[1]{\textsc{#1}}
\usepackage{url}
\usepackage[normalem]{ulem}
\newcommand{\ul}[1]{\uline{#1}}
\newcommand{\txx}[1]{\textbf{#1}}

\begin{document}
\maketitle

\begin{enumerate}
  
\item Find at least one example each of words with positive and
  negative sentiment from your story.
  \begin{itemize}
  \item Do this ahead of class
  \item Note the entire sentence
  \end{itemize}
   \emph{In the morning the captain returned; he was \ul{angry} and \ul{unhappy}, but said nothing.} (negative, negative)
  \\ \emph{Fred is my \ul{best friend} after all.} (positive)


\item \label{q:valid-argument} Which of the following arguments is \textbf{valid}?
\begin{enumerate}
\item If it rains, the ground gets wet. It is raining. Therefore, the ground gets wet. \label{q:valid-argument-a}
\item If it rains, the ground gets wet. The ground is wet. Therefore, it must have rained.
\item If it rains, the ground gets wet. It is sunny. Therefore, the ground is wet.
\end{enumerate}
\abox{Answer: (\ref{q:valid-argument-a})}

\item \label{q:true-statement} Which statement must be \textbf{true}?
\begin{enumerate}
\item All dogs are mammals. \label{q:true-statement-a}
\item Some dogs are reptiles.
\item All mammals are dogs.
\end{enumerate}
\abox{Answer: (\ref{q:true-statement-a})}

\item \label{q:inclusive-or} In logic, "\textbf{or}" (inclusive) means:
\begin{enumerate}
\item One or the other, but not both.
\item Either one or both. \label{q:inclusive-or-b}
\item Neither one nor the other.
\end{enumerate}
\abox{Answer: (\ref{q:inclusive-or-b})}

\newpage
  
\item Are the following quantifiers:
  \begin{itemize}
  \item[(i)] \textbf{Symmetrical or asymmetrical}?  
  (Symmetrical means switching subject and predicate does not change truth; asymmetrical means it does.)
  \item[(ii)] How do they behave with entailments?
    \begin{itemize}
    \item[(iib)] In the \textbf{left argument} (the restrictor): is the quantifier upward or downward entailing?  
    (Upward entailment means moving from a subset to a superset preserves truth; downward entailment means moving from a superset to a subset preserves truth.)
    \item[(iic)] In the \textbf{right argument} (the scope): is the quantifier upward or downward entailing?  
    (Again, upward entailment means weakening the predicate preserves truth; downward entailment means strengthening it preserves truth.)
    \end{itemize}
  \end{itemize}
\begin{exe}
\ex \textit{most}
     \abox{\begin{itemize}
       \item Asymmetrical (e.g., Most cats are animals ≠ Most animals are cats).
       \item Left downward entailing: Most black cats are animals → Most cats are animals is NOT valid (subset in restrictor).
       \item Right upward entailing: Most cats meow → Most cats make noise IS valid (weaken predicate).
     \end{itemize}}
\ex \textit{many (cardinal)}
     \abox{\begin{itemize}
       \item Asymmetrical (e.g., Many cats are friendly ≠ Many friendly are cats).
       \item Left upward entailing: Many black cats are friendly ⟶ Many cats are friendly.
       \item Right upward entailing: Many cats meow ⟶ Many cats make noise.
     \end{itemize}}
\ex \textit{few (cardinal)}
     \abox{\begin{itemize}
       \item Asymmetrical (e.g., Few animals are cats ≠ Few cats are animals).
       \item Left downward entailing: Few animals are cats ⟶ Few animals are black cats.
       \item Right downward entailing: Few cats make noise ⟶ Few cats meow.
     \end{itemize}}
\ex \textit{every}
     \abox{\begin{itemize}
       \item Asymmetrical (e.g., Every cat is an animal ≠ Every animal is a cat).
       \item Left downward entailing: Every cat is an animal ⟶ Every black cat is an animal
       \item Right upward entailing: Every cat meows ⟶ Every cat makes noise.
     \end{itemize}}
\ex \textit{at least two}
     \abox{\begin{itemize}
       \item Asymmetrical (e.g., At least two cats are animals ≠ At least two animals are cats).
       \item Left upward entailing: At least two black cats are friendly ⟶ At least two cats are friendly.
       \item Right upward entailing: At least two cats meow ⟶ At least two cats make noise.
     \end{itemize}}
\ex \textit{exactly two}
     \abox{\begin{itemize}
       \item Symmetrical (e.g., Exactly two cats are animals ≈ Exactly two animals are cats, assuming sets match).
       \item Neither clearly upward nor downward entailing: exact number depends on both sides.
       \item Caution: changing restrictor or scope may disrupt the exact count.
     \end{itemize}}
 \end{exe}



 
\end{enumerate}
\end{document}

%%% Local Variables: 
%%% coding: utf-8
%%% mode: latex
%%% TeX-PDF-mode: t
%%% TeX-engine: xetex
%%% End: 
