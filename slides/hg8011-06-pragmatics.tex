\PassOptionsToPackage{xetex}{xcolor}
\PassOptionsToPackage{xetex}{graphicx}
\documentclass[a4paper,landscape,headrule,footrule,xetex]{foils}

%%
%%%  Macros
%%%
\newcommand{\logo}{~}
\newcommand{\Story}{\SHA{FINA}{The Final Problem}}

\newcommand{\header}[3]{%
\title{\vspace*{-2ex} \Large DAS
\\\large  Semantics
\\[2ex] \Large  \emp{#2}}
\author{\blu{Francis Bond}   \\ 
  \normalsize  \textbf{Department of Asian Studies}\\
   \normalsize  \textbf{Palacký University}\\
\normalsize  \url{https://fcbond.github.io/}\\
\normalsize  \texttt{bond@ieee.org}}
\date{#1}
\renewcommand{\logo}{#2}
 \hypersetup{
   pdfinfo={
     Author={Francis Bond},
     Title={#1 #2},
     Subject={DAS Semantics},
     Keywords={Semantics, Pragmatics},
     License={CC BY 4.0}
   }
 %  pdfcopyright={Copyright © Francis Bond. Creative Commons 4.0 Attribution License.}
 %  pdflicenseurl={http://creativecommons.org/licenses/by/4.0/}
 }
}
%%
%% Multilingual Stuff
%%
\usepackage[a4paper,landscape,margin=25mm]{geometry}

\MyLogo{Semantics (2024); CC BY 4.0}



\usepackage{fontenc}
\usepackage{polyglossia}
\setmainlanguage{english}
\setmainfont{TeX Gyre Pagella}
\setsansfont[Ligatures=TeX]{TeX Gyre Heros}
\usepackage{xeCJK}
\setCJKmainfont{Noto Sans CJK SC}
\setCJKsansfont{Noto Sans CJK SC}
\setCJKmonofont{Noto Sans CJK SC}
%\setCJKttfont{Noto Sans CJK SC}
%\setCJKmainfont{WenQuanYi Micro Hei}
%\clearpage
%\setCJKmainfont{AR PL SungtiL GB}
\newfontfamily\ipafont{Charis SIL}
\newcommand\ipa[1]{\mtcitestyle{\ipafont #1}}


\usepackage[xetex]{xcolor}
\usepackage[xetex]{graphicx}
\newcommand{\blu}[1]{\textcolor{blue}{#1}}
\newcommand{\grn}[1]{\textcolor{green}{#1}}
\newcommand{\hide}[1]{\textcolor{white}{#1}}
\newcommand{\emp}[1]{\textcolor{red}{#1}}
\newcommand{\txx}[1]{\textbf{\textcolor{blue}{#1}}}
\newcommand{\lex}[1]{\textbf{\mtcitestyle{#1}}}

\usepackage{pifont}
\renewcommand{\labelitemi}{\textcolor{violet}{\ding{227}}}
\renewcommand{\labelitemii}{\textcolor{purple}{\ding{226}}}

\newcommand{\subhead}[1]{\noindent\textbf{#1}\\[5mm]}

\newcommand{\Bad}{\emp{\raisebox{0.15ex}{\ensuremath{\mathbf{\otimes}}}}}
\newcommand{\bad}{*}

\newcommand{\com}[1]{\hfill \textnormal{(\emp{#1})}}%
\newcommand{\cxm}[1]{\hfill \textnormal{(\txx{#1})}}%
\newcommand{\cmm}[1]{\hfill \textnormal{(#1)}}%
\usepackage{amssymb}
\usepackage{relsize,xspace}
\newcommand{\into}{\ensuremath{\rightarrow}\xspace}
\newcommand{\ent}{\ensuremath{\Rightarrow}\xspace}
\newcommand{\nent}{\ensuremath{\not\Rightarrow}\xspace}
\newcommand{\tot}{\ensuremath{\leftrightarrow}\xspace}
\usepackage{url}
\usepackage[hidelinks]{hyperref}
\hypersetup{
     colorlinks,
     linkcolor={blue!50!black},
     citecolor={red!50!black},
     urlcolor={blue!80!black}
}
%\usepackage{hyperxmp}
\newcommand{\lurl}[1]{\MyLogo{\url{#1}}}

\usepackage{mygb4e}
\let\eachwordone=\itshape
\newcommand{\lx}[1]{\textbf{\textit{#1}}}
\newcommand{\ix}{\ex\it}

\newcommand{\cen}[2]{\multicolumn{#1}{c}{#2}}
%\usepackage{times}
%\usepackage{nttfoilhead}
\newcommand{\myslide}[1]{%
\foilhead[-25mm]{\raisebox{12mm}[0mm]{\emp{#1}}}%
\leftheader{}%
\MyLogo{\logo}}

\newcommand{\mytask}[1]{%
\foilhead[-25mm]{\raisebox{12mm}[0mm]{\emp{#1}}}
\leftheader{🔍 Hi}%
\MyLogo{\logo}}

\newcommand{\myslider}[1]{\rotatefoilhead[-25mm]{\raisebox{12mm}[0mm]{\emp{#1}}}}
%\newcommand{\myslider}[1]{\rotatefoilhead{\raisebox{-8mm}{\emp{#1}}}}

\newcommand{\section}[1]{\myslide{}{\begin{center}\Huge \emp{#1}\end{center}}}

\usepackage{tcolorbox}
% \newcommand{\task}{\marginpar{\raisebox{-1ex}{\large
%       \tcbox[colframe=red,colback=white,arc=3pt]{\textbf{?}}}}}
% \newcommand{\task}{\marginpar{\raisebox{-1ex}{
%       \hspace{-0.5em}\tcbox[colframe=red,colback=white,arc=3pt]{%
%         \includegraphics[width=1.5em]{pics/detective}}}}}
\newcommand{\task}{\marginpar{\raisebox{-2ex}{
      \hspace{-0.5em}\reflectbox{\includegraphics[width=2em]{pics/detective}}}}}

\usepackage[lyons,j,e,k]{mtg2e}
\renewcommand{\mtcitestyle}[1]{\textcolor{teal}{\textsl{#1}}}
%\renewcommand{\mtcitestyle}[1]{\textsl{#1}}
\newcommand{\chn}{\mtciteform}
\newcommand{\cmn}{\mtciteform}
\newcommand{\cs}{\mtciteform}
\newcommand{\iz}[1]{\textup{\texttt{\textcolor{blue}{\textbf{#1}}}}}
\newcommand{\con}[1]{\textsc{#1}}
\newcommand{\gm}{\textsc}
\newcommand{\cmp}[1]{{[\textsc{#1}]}}
\newcommand{\sr}[1]{\ensuremath{\langle}#1\ensuremath{\rangle}}
\usepackage[normalem]{ulem}
\newcommand{\ul}{\uline}
\newcommand{\ull}{\uuline}
\newcommand{\wl}{\uwave}
\newcommand{\vs}{\ensuremath{\Leftrightarrow}~}
%%%
%%% Bibliography
%%%
\usepackage{natbib}
%\usepackage{url}
\usepackage{bibentry}


%%% From Tim
\newcommand{\WMngram}[1][]{$n$-gram#1\xspace}
\newcommand{\infers}{$\rightarrow$\xspace}



\usepackage{rtrees,qtree}
\renewcommand{\lf}[1]{\br{#1}{}}
\usepackage{avm}
%\avmoptions{topleft,center}
\newcommand{\ft}[1]{\textsc{#1}}
%\newcommand{\val}[1]{\textit{#1}}
\newcommand{\typ}[1]{\textit{#1}}
\avmfont{\sc}
%\avmvalfont{\sc}
\renewcommand{\avmtreefont}{\sc}
\avmsortfont{\it}


%%% From CSLI book
\newcommand{\mc}{\multicolumn}
\newcommand{\HD}{\textbf{H}\xspace}
\newcommand{\el}{\< \>}
\makeatother
\long\def\smalltree#1{\leavevmode{\def\\{\cr\noalign{\vskip12pt}}%
\def\mc##1##2{\multispan{##1}{\hfil##2\hfil}}%
\tabskip=1em%
\hbox{\vtop{\halign{&\hfil##\hfil\cr
#1\crcr}}}}}
\makeatletter

\newcommand{\sh}[1]{\lowercase{\href{https://fcbond.github.io/sh-canon/#1.html}}{#1}}
\newcommand{\SHA}[2]{\lowercase{\href{https://fcbond.github.io/sh-canon/#1.html}}{\textit{#2}}}

\usepackage{tikzsymbols}
\newcommand{\PF}[1]{\Smiley[#1][green]}
\newcommand{\NF}[1]{\Annoey[#1][red]}


\begin{document}
\header{Lecture 6}{Pragmatics and Discourse}{}
\maketitle

%\include{schedule}


\myslide{Overview}

\begin{itemize}
\item Review
\item Context
\item Conversation and Cooperation
\item Conversational Maxims
\item Politeness
\item Indirect Speech
\item Summary

\end{itemize}

\section{Revision}

\myslide{Idioms and Metaphors}
\begin{itemize}
\item Many phrases have meanings that cannot be predicted from the meanings of the individual words
  \begin{itemize}
  \item \eng{take into one's confidence}
  \item \eng{take in}
  \item \eng{Sherlock Holmes}
  \item \eng{practical joke(r)}
  \item \eng{in love}
  \end{itemize}
\item Metaphors extend the use of words beyond their primary meaning
  to describe referents that bear similarities to the word's primary
  referent.
\begin{exe}
  \ex \eng{"Oh, sir, do you not think that you could help me, too, and  and at least \ul{throw a little light} through \ul{the dense darkness which surrounds me}"}
\end{exe}
\begin{center}
UNDERSTANDING is LIGHT \\ or \\ IGNORANCE is DARKNESS
\end{center}
\end{itemize}


\section{Reference and Context}
\MyLogo{}

\myslide{We interpret words in \txx{context}}

\begin{itemize}\addtolength{\itemsep}{-1.5ex}
%\item It isn't only deictic expressions that require context
\item For example, in a bookshop
  \begin{exe}
    \ex \eng{Have you got the new \ul{C.J. Cherryh}?} \textnormal{``book by $\sim$''}
  \end{exe}
\item In a snooker (pool)  game
  \begin{exe}
    \ex \eng{I have two \ul{reds} left} ``red balls''
  \end{exe}
\item \txx{metonymy}: substituting the name of an attribute or feature for the name of the thing itself
  \begin{exe}
    \ex \eng{\ul{The ham sandwich} is at table three} ``person who ordered $\sim$''
    \ex \eng{I spent all morning with \ul{the suits}} ``person who
    habitually wears $\sim$''\footnote{ A person who wears matching
      jacket and trousers, especially a boss or a supervisor (pejorative)}
  \end{exe}
  \item \txx{synecdoche}: substituting the name of a part for the name of a thing \\
    (a kind of metonymy)
  \begin{exe}
    \ex \eng{We need some more willing \ul{hands}} ``person with $\sim$''
  \end{exe}
\item[?] Give examples of metonymy and synechdoche\task

\end{itemize}


\myslide{All knowledge is context}
\begin{itemize}\addtolength{\itemsep}{1.5ex}
\item Knowledge to interpret utterances can come from multiple sources
  \begin{enumerate}\addtolength{\itemsep}{1ex}
  \item \txx{Deixis}: The physical context of the utterance
    \begin{itemize}
    \item \eng{My stepdaughter has been \ul{here}. I have traced her.} ``221B Baker Street'' \sh{SPEC}
    \end{itemize}

  \item \txx{Discourse}: What has already been said
    \begin{itemize}
    \item \eng{The dog$_i$ chased the cat$_j$.  Eventually it$_i$ caught it$_j$.}
    \item \eng{My stepdaughter$_i$ has been here. I have traced her$_i$.} \sh{SPEC}
    \end{itemize}
  \item \txx{World knowledge}: Background and common knowledge
   \begin{itemize}
    \item \eng{I would like to go to the moon.} ``the Earth's moon''
    \end{itemize}

  \end{enumerate}
\end{itemize}

\myslide{Context can complete fragments}
\begin{itemize}
\item In a dialogue, we often only add new knowledge
  \begin{exe}
    \ex 
    \begin{xlist}
      \ex \eng{Who moved these chairs?}
      \ex \eng{Sandy (did) [move these chairs]}
  \end{xlist}
    \ex 
    \begin{xlist}
      \ex \eng{Where are you going?}
      \ex \eng{[I am going] (to) Tokyo}
  \end{xlist}
  \ex
  \begin{xlist}
   \ex \eng{'What is it, then? [Is it] A fire?'}
   \ex \eng{'No [it is not], [It is] a client.  \ldots} \hfill \sh{SPEC}
 \end{xlist}
\end{exe}
\item Normally English requires a complete sentence, 
\item[\ldots] but here a \txx{fragment} is OK
\end{itemize}


% If you are interested in this, NTU's LMS is a center for \txx{conversational analysis}.

% \myslide{Discourse Topic}
% \MyLogo{\url{http://lessonstream.org/2011/04/30/washing-clothes/}}
% \begin{itemize}
% \item It is much easier to understand an utterance if you know what
%   it is about \citep{Bransford:Johnson:1972}

% \item Giving the same text different titles changes your interpretation

% \item Giving a discourse topic aids in understanding and retention

% \end{itemize}

% \myslide{Background Knowledge}

% What knowledge do we need to interpret the following?

% \begin{exe}
%   \ex 
%   \begin{xlist}
%     \ex \eng{I'm hungry}
%     \ex \eng{I'll lend you some money}
%   \end{xlist}
%   \ex 
%   \begin{xlist}
%     \ex \eng{Shall we get some icecream?}
%     \ex \eng{I'm on a diet}
%   \end{xlist}
%   \ex 
%   \begin{xlist}
%     \ex \eng{Shall we lunch next week?}
%     \ex \eng{It's Ramadan}
%   \end{xlist}
%  \ex 
%   \begin{xlist}
%     \ex \eng{Kim chased the dog with a stick}
%     \ex \eng{Kim chased the dog with a bone}
%     \ex \eng{Kim chased the dog with a broom}
%     \ex \eng{Kim chased the dog with a white tail}
%     \ex \eng{Kim chased the dog with a wound}

%   \end{xlist}

% \end{exe}

% %\myslide{Mutual Knowledge}

% \myslide{Formalizing Knowledge for Computers}

% \begin{itemize}
% \item There is a lot of work on making knowledge available to
%   computers so that they can interpret text
%   \begin{itemize}
%   \item Formal ontologies (knowledge-based)
%     \begin{itemize}
%     \item Scripts
%     \item Wordnets
%     \item CYC
%     \end{itemize}
%   \item Example-based (compare to existing examples)
%     \\ collect many fragments of existing knowledge
%   \end{itemize}
% \end{itemize}


% \myslide{What can computers do?}
% \MyLogo{}
% \begin{itemize}
% \item[Q] \eng{Do birds migrate through Turkey?}
% \item[A] Yes.  \eng{The crane ($\subset$ bird) flies across Ankara ($\subset_{in}$ Turkey)}.
%   \begin{quote}
%     \large   fly$_1$ ($e_1$, crane$_5$($x_1$)),  across($e_2$,$e_1$,$x_2$), Ankara($x_2$).
%   \end{quote}
%   \begin{itemize}
%   \item  \eng{through} and \eng{across} are both \textit{path} roles.
%   \item \eng{fly} and \eng{migrate} are both \textit{motion} verbs.
%   \item Ankara is in Turkey
%   \end{itemize}
% \item Why do this?
%   \begin{itemize}
%   \item Local environmental knowledge is often not translated into many languages
%   \item Facts may only be recorded in a few documents
% \end{itemize}
% \end{itemize}



% \section{Inference}

% \myslide{Anaphoric Reference}
% \MyLogo{In bridging, the referent is different, but linked in some way.}
% \begin{itemize}
% \item A pronoun (or definite nominal) can refer back to something
%   earlier in the discourse
%   \begin{exe}
%     \ex 
%     \begin{xlist}
%     \ex \eng{I tripped over a dog. \ul{The dog} bit me.}
%     \ex \eng{I tripped over a dog. \ul{The beast} bit me.}
%     \ex \eng{I tripped over a dog. \ul{It} bit me.}
%     \ex \eng{I tripped over a dog. \ul{The tail} tangled me.} \hfill \txx{bridging}
%     \ex \eng{I tripped over a dog. $\phi$ bit me.} 
%   \end{xlist}
%   \ex 
%  \begin{xlist}
%    \ex \eng{I left early.  I had a train to catch.}
%    \trans \textnormal{Inference: I left early \textbf{because} I had a train to catch.}
%  \end{xlist}
% \end{exe}
% \item People use inference to 
%   \begin{itemize}
%   \item Interpret pronouns and nominals
%   \item More generally, to link information together
%   \end{itemize}
% \end{itemize}
 
\section{Conversational Implicature}


\myslide{Cooperation in Conversation}
\MyLogo{\citet{Grice:1975}}
\begin{itemize}
\item \txx{Cooperative Principle}: people cooperate in conversation
  \begin{quote}
    ``Make your conversational contribution such as is required, at the stage at which it occurs, by the accepted purpose or direction of the talk exchange in which you are engaged.''
  \end{quote}
\item \txx{Implicature}
  \begin{quote}
    The aspect of meaning that a speaker conveys, implies, or suggests
    without directly expressing.
  \end{quote}
  \begin{exe}
    \ex \eng{Did you do the reading?}
    \ex \eng{I meant to.}
    \trans \textnormal{Implicates: No}
  \end{exe}

\end{itemize}

\myslide{Gricean Maxims}
\MyLogo{\citet{Grice:1975}}
\begin{description}
\item [\txx{Maxim of Quantity}] ~
  \begin{itemize}
  \item Make your contribution as informative as is required (for the current purposes of the exchange).
  \item Do not make your contribution more informative than is required.
  \end{itemize}
\item [\txx{Maxim of Quality}] ~
  \begin{itemize}
  \item Do not say what you believe to be false.
  \item Do not say that for which you lack proper evidence.
  \end{itemize}
\newpage
\item [\txx{Maxim of Relation}] ~
  \begin{itemize}
  \item Be relevant.
  \end{itemize}
\item [\txx{Maxim of Manner}] ~
  \begin{itemize}
  \item Be perspicuous [= be easily understood]\footnote{A
      philosopher's joke.}
  \item Avoid obscurity of expression.
  \item Avoid ambiguity
  \item Be brief (avoid unnecessary prolixity)
  \item Be orderly
  \end{itemize}
\end{description}

\myslide{An Example of implicature}

Speech that seems to violate the maxims will evoke \txx{implicatures}
(inferences about the reason why the speaker violated the maxim(s)).
This is because the hearer assumes the speaker is acting in accordance
with the Cooperative Principle, and is rational.
\begin{exe}
  \ex A: \eng{Can you tell me the time?}
  \trans Lit: Do you have the ability to tell me the time?
  \ex B: \eng{Well, the milkman has come.}
  \trans Lit.: The milkman came at some time prior to the time of speaking.
\end{exe}
\newpage
What is meant:
\begin{itemize}
\item [A] Do you have the ability to tell me the time of the present moment, as standardly indicated on a watch, and if so, please do so tell me what time it is.
\item [B] No, I don't know the exact time of the present moment, but I can provide some information from which you may be able to deduce the approximate time, namely the milkman, who delivers milk at 6:30am,  came at some time prior to the time of speaking.
\end{itemize}

\begin{itemize}
\item [A] flouted Manner --- why not request that you are told the time?
\item [B] flouted Relation --- what does this have to do with the time?
\end{itemize}

\myslide{Various Conversational Implicatures}
\begin{itemize}
\item Sometimes no special knowledge is required in the context to
  calculate the additional conveyed meaning 
  (\txx{Generalized Conversational Implicatures})
  \begin{exe}
    \ex \eng{Did you bring the flowers and the card?}
    \ex \eng{I brought the card.}
    \trans \textnormal{Implicature: but not the flowers.}
  \end{exe}
\item 
Most of our conversations take place in very specific contexts in
which locally recognized inferences are assumed. 
(\txx{Particularized Conversational Implicatures})

\begin{exe}
  \ex \eng{Hey Terry, are you coming to the party tonight?}
  \ex \eng{My parents are visiting.} ``So I am busy/So I have a babysitter''
\end{exe}
\item All implicatures are \txx{defeasible}: they can be canceled
without a contradiction.

\begin{exe}
  \ex \eng{But I can still come.}
\end{exe}
\end{itemize}



\myslide{Scalar Implicatures}
\MyLogo{Extended into the Horn Scale: classic example is numbers}
\newcommand{\horn}[1]{\ensuremath{\langle}\,#1\,\ensuremath{\rangle}}
\newcommand{\set}[1]{\{#1\}}

Certain information is communicated by choosing a word which
expresses one value from a scale of values. 

\begin{exe}
  \ex	$\langle$ \eng{all, most, many, some, few} $\rangle$
  \ex	$\langle$ \eng{always, often, sometimes} $\rangle$
\end{exe}

We should choose the word from the scale which is the most informative
and truthful in the circumstances (Quantity and Quality).  Words on
the scale implicate the negation of words on their left: 

\begin{exe}
  \ex \eng{I'm doing a major in Linguistics and I've completed some of the required subjects}
  \ex \eng{They are often late.}
  \ex	\eng{I got some of these antiques in London – hang on, actually I think I got most of them 	there.} \hfill (defeasible)
\end{exe}

\myslide{Horn Scales}
%\MyLogo{\url{http://www.personal.uni-jena.de/~mu65qev/wikolin/index.php?title=Horn_scale&redirect=no}}
 To form a
Horn scale \horn{$S,W$}, two words ($S$ and $W$) must satisfy the following
conditions:
\begin{enumerate}\addtolength{\itemsep}{-2ex}
\item[(i)] $A(S)$ must entail $A(W)$ for some arbitrary sentence frame $A$;
\item[(ii)] $S$ and $W$ must be equally lexicalized;
\item[(iii)] $S$ and $W$ must be about the same semantic relations, or
  from the same semantic field. 
\end{enumerate}
  \begin{itemize}
\item Words on the scale implicate the negation of words on their left
  \begin{itemize}
  \item \horn{\eng{always, often, sometimes}}.
  \item \horn{\eng{\ldots, 5, 4, 3, 2, 1}}.
  \item \horn{\eng{hot, warm, lukewarm, cold}}.
  \item \horn{\eng{the},  \set{\eng{a},\eng{some}}}.
  \end{itemize}
\end{itemize}



% \myslide{Conventional Implicatures}
% \MyLogo{}

% Conventional implicatures are non-truth conditional inferences that
% are not derived from superordinate pragmatic principles like the
% [Gricean] maxims, but are simply attached by convention to particular
% lexical items.

% They are non-cancellable:
% \begin{exe}
%   \ex
%   \begin{xlist}
%     \ex \eng{She was poor, but honest.}
%     \ex \eng{*She was poor but honest, and was in fact rich.}
%   \end{xlist}
% \end{exe}

\myslide{Flouting the maxims}
\begin{itemize}
\item Quantity:	(In answer to \eng{Tell me all about him!}:) 
  \eng{He has a nice personality.}
\item Quality:	(In response to something stupid someone did:) 
  \eng{That was brilliant!}
\item Relation:	(In response to \eng{Can I go out and play?}:) 
\eng{Did you finish your homework?}
\item Quality:
  \begin{exe}
    \ex \eng{My car breaks down every five minutes} \hfill hyperbole
    \ex \eng{I've got millions of bottles of wine in my cellar} \hfill hyperbole    \ex \eng{Queen Victoria was made of iron} \hfill metaphor
    \ex \eng{I love it when you sing out of tune} \hfill irony or sarcasm
  \end{exe}
  
\end{itemize}
%% FIXME add some manner flouting

\myslide{What happens when we flout?}
\begin{itemize}
\item If someone doesn't understand this, (e.g. someone from another
  culture), then what was originally intended to be a metaphor may
  result in a \txx{lie}.
\item We may flout: 
  \begin{itemize}
\item Quantity: 
  \begin{itemize}
  \item say  more than we need to indicate a sense of occasion, or respect
  \item  say less than we need, in order to be blunt, or rude
  \end{itemize}
\item Quality: 
  \begin{itemize}
  \item white lies
  \end{itemize}
\item Relation  
  \begin{itemize}
  \item to signal embarrassment
  \item to change the subject
  \end{itemize}
\item Manner
  \begin{itemize}
  \item for the   sake of humour
  \item to obscure information (parents talking in front of children)
  \item to show in-group status,  \ldots
  \end{itemize}
\end{itemize}
\end{itemize}


\myslide{Hedges}

When we \txx{flout}  a maxim, we can use \txx{hedges}:

\begin{exe}
\ex Quantity:
    \begin{xlist}
    \ex \eng{\ul{As you probably know},} \ldots
    \ex \eng{\ul{To cut a long story short},} \ldots
  \end{xlist}
\ex Quality:
  \begin{xlist}
    \ex \eng{In the kitchen, \ul{I believe}.}\hfill \sh{DANC}
    \ex \eng{\ul{As far as I'm aware}, Kim is still on medication.}
  \end{xlist}
\ex Relation:
    \begin{xlist}
    \ex \eng{\ul{I don't know if this will affect the bottom line}, but some of the numbers are  missing.}
  \end{xlist}
  % b. This may seem beside the point, but we can't ignore the impact of lower interest
  %  rates on these investments.
  %       c. Not that I'm changing the topic, but is this related to the budget?

\ex Manner:
 \begin{xlist}
    \ex \eng{\ul{I'm not sure if this makes sense}, but the car had no lights.}
  \end{xlist}
\end{exe}

\myslide{Your turn, I guess}

Hedges are used when you know you will flout a maxim.  Which
  maxim is flouted in the following hedges (and why)?\task

\begin{exe}
% Manner:
\ex \eng{This may be a bit confused, but I remember being in a car.}

%Quality:
\ex \eng{I may be mistaken, but I thought I saw a wedding ring on her finger.}

%\ex \eng{ I'm not sure if this is right, but I heard it was a secret ceremony in Hawaii.
%\ex \eng{ He couldn't live without her, I guess.

% Quantity:
%\ex \eng{  So, to cut a long story short, we grabbed our stuff and ran.
\ex \eng{I won't bore you with all the details, but it was an exciting trip.}

% Relation:
\ex \eng{I don't know if this is important, but some of the files are missing.}

%Quality:
\ex \eng{As far as I know, they're married.}

% Relation:
\ex \eng{This may sound like a dumb question, but whose handwriting is this?}
%\ex \eng{  Not to change the subject, but is this related to the budget?

% Manner:
%\ex \eng{I'm not sure if this makes sense, but the car had no lights.
\ex \eng{I don't know if this is clear at all, but I think the other car was reversing.}
% Quantity:
\ex \eng{As you probably know, I am afraid of dogs.}
\end{exe}

\section{Politeness}

\myslide{Why be Indirect?}

\begin{itemize}
\item Mainly for politeness

  \begin{exe}
    \ex {[Motorist to gas station attendant]}
    \begin{xlist}
      \ex \eng{You don't happen to have any change for the phone do you?}
    \end{xlist}
    \ex {[Doctor to Nurse]}
    \begin{xlist}
      \ex \eng{I'll need a 19 gauge needle, IV tubing and some unobtanium}
    \end{xlist}
  \ex {[Teacher to student?]}
    \begin{xlist}
      \ex \eng{Would you be so kind as to give me a hand with this?}
    \end{xlist}
  \end{exe}
\item[$\Rightarrow$] Low Status \into High Status is generally more indirect than High \into Low 
\end{itemize}

\myslide{Politeness and Face-Threatening Acts}
\MyLogo{\citet{Brown:Levinson:1987}}
\begin{itemize}
\item[\PF{2}] \txx{Positive Face} desire to seem worthy and deserving of approval
  \\ \emp{self-worth}: I want you to like me!
\bigskip\bigskip\bigskip
\item[\NF{2}] \txx{Negative Face} desire to be autonomous, unimpeded by others
  \\ \emp{freedom}: I want you not to bother me!
\bigskip\bigskip\bigskip
\item It is argued that we all have these two faces --- they are universal
\item But they are always under threat!

\end{itemize}

\myslide{Face Threatening Acts}

\begin{itemize}
\item[\PF{1}] Threaten Positive Face
  \begin{itemize}
  \item Hearer
    \begin{itemize}
    \item explicit expressions of disapproval 
    \item expressions of indifference, interruption, boasting
    \item identification of status (\eng{boy} not \eng{doctor})
    \end{itemize}
  \item Speaker
    \begin{itemize}
    \item apologies, accepting a compliment, confession, losing control
    \end{itemize}
  \end{itemize}
\item[\NF{1}] Threaten Negative Face
  \begin{itemize}
  \item Hearer
    \begin{itemize}
    \item orders, requests, suggestions, advice
    \item compliments, expressions of envy or admiration
    \item offers or promises (adds obligation)
    \end{itemize}
  \item Speaker
    \begin{itemize}
    \item thanks, excuses, acceptance of offers or apologies
    \end{itemize}
  \end{itemize}
\end{itemize}

\myslide{Face Saving Strategies} \MyLogo{Choice depends on social
  distance, power asymmetry, nature of the act}
  \begin{itemize}\addtolength{\itemsep}{-1ex}
\item Bald (on-record)
\item Positive Politeness: 
  \begin{itemize}
  \item be attentive, appeal to in-group, joke 
  \item reciprocate: \eng{I'll help you if you help me}
  \item compliment: \eng{You're looking good today, \ldots}
  \end{itemize}
\item Negative Politeness: 
  \begin{itemize}
  \item hedge to minimize threat: \eng{I may be wrong but, \ldots}
  \item allow for negative face: \eng{Could you please, \ldots}
  \item ask indirectly: \eng{Have you got the time, \ldots}
  \end{itemize}
\item Indirect (off-record)
  \begin{itemize}
  \item \eng{It's hot in here} ``please turn on the aircon''
  \end{itemize}
\item[?] Which face is threatened, and how does Holmes save it?
  \eng{There may be some little danger, so kindly put your army revolver in your pocket.'} \hfill \sh{REDH}\task
\end{itemize}



\myslide{An example of polite, indirect speech (gone wrong)}
\MyLogo{The Big Bang Theory: \href{https://www.youtube.com/watch?v=fhv1dOae9MUAn}{The Apology Insufficiency (S4E7)}}

\begin{exe}
  \ex {[Knock on the door]}
  \ex Leonard: \eng{Wanna get that?}
  \ex Sheldon: \eng{Not particularly.}
  \ex Leonard: \eng{Could you get that?}
  \ex Sheldon: \eng{I suppose I could if I were asked.}
  \trans {[Knock on the door]}
  \ex Leonard: \eng{Would you please get that?}
  \ex Sheldon: \eng{ Well of course! \\  Why do you have to make things so complicated?}
\end{exe}



\section{Austin's Speech Act Theory}

\myslide{Speech as Action}
\MyLogo{\citet{Austin:1962}}
\begin{itemize}
\item Language is often used to \emp{do} things: \txx{speech acts}
  \\ language has both
  \begin{itemize}
  \item \txx{interactivity}
  \item \txx{context dependence}
  \end{itemize}
\item E.g. If you greet someone or ask them a question, and they don't
  respond it is very awkward
\end{itemize}

\myslide{Sentence Types}
\MyLogo{A bit like tense and time}
\begin{itemize}
\item There are four syntactic types that correlate closely to pragmatic uses

  \begin{tabular}{lcll}
    syntactic type &  & pragmatic use & example \\ \hline
  \txx{declarative}  &$\leftrightarrow$& \txx{assertion} & \eng{This is my friend}\\
  \txx{interrogative} &$\leftrightarrow$& \txx{question}  & \eng{Are you my friend?}\\
  \txx{imperative} &$\leftrightarrow$& \txx{order}  & \eng{Be my friend!}\\
  \txx{optative} &$\leftrightarrow$& \txx{wish} & \eng{Oh that you were my friend!}
  \end{tabular}
\item But it turns out there is a lot of flexibility:
  \begin{exe}
    \ex \begin{xlist} 
      \ex \eng{Would you like a beer?}\hfill question
      \ex \eng{Is the pope Catholic?} \hfill assertion
      \ex \eng{You are sure that she has not sent it yet?} (\sh{SCAN})
       \hfill question

    \end{xlist}
  \end{exe}
\end{itemize}

\myslide{Language as Truth}
\begin{itemize}
\item One tradition of semantics is based on these assumptions
  \begin{itemize}
  \item the basic sentence type is declarative
  \item language is mainly used to describe the world
  \item meaning can be given in terms of truth values
  \end{itemize}
\item It doesn't deal well with these
  \begin{exe}
    \ex \eng{Excuse me!}
    \ex \eng{Hello.}
    \ex \eng{How much can a Koala bear?}
    \ex \eng{Six pints of lager and some nachos, thanks!}
    \ex \eng{How 'bout them niners?}
  \end{exe}
\end{itemize}

\myslide{Perfomative Utterances}

\begin{exe}
  \ex \eng{I \ul{promise} I won't drive home}
  \ex \eng{I \ul{bet} you 5 bucks they get caught }
  \ex \eng{I \ul{declare} this lecture over} 
  \ex \eng{I \ul{warn} you that legal action will ensue}
  \ex \eng{I \ul{name} this ship \ul{the Nautilus}}
\end{exe}

\begin{itemize}
\item Uttering these (in an appropriate context) \emp{is} acting
\\  \emp{Utterances themselves can be actions}
\item In English, we can signal this explicitly with \lex{hereby}
\end{itemize}

\myslide{Felicity Conditions}
\MyLogo{\citep{Austin:1962}}
\begin{itemize}
\item Performatives (vs Constantives) 
\\ Given the correct \txx{felicity conditions}
  \begin{description}
  \item[A1] There must exist an accepted conventional procedure that
    includes saying certain words by certain persons in certain
    circumstances,
  \item[A2] The circumstances must be appropriate for the invocation
  \item[B1] All participants must do it both correctly
  \item[B2] \ldots  and completely
  \item[C1] The intention must be to do this the act
  \item[C2] The participants must conduct themselves so subsequently.
  \end{description}
\item If the conditions don't hold, the speech act is \txx{infelicitous}
  \begin{itemize}
  \item Failing \textbf{A} or \textbf{B} is a \txx{misfire}
  \item Failing \textbf{C} is an \txx{abuse}
  \end{itemize}
  \end{itemize}

\myslide{Examples of Infelicities}
\begin{itemize} \addtolength{\itemsep}{-1ex}
\item \textbf{A1} \eng{I hereby marry you} (said by someone not authorized to do so)
\item \textbf{A2} \eng{I baptize this baby Harold} (baby's name should
  be Herman)
\item \textbf{A2} \eng{I pronounce John Smith dead} (uttered by a doctor who has confused John Smith
with John Smit, or if John Smith is still alive)
\item \textbf{B1}
\eng{Yes} (exchanging vows in a Christian marriage ceremony)
\item \textbf{B1}
\eng{OK} (in response to \eng{Do you swear to tell the truth, the whole truth and nothing but
the truth?} – wrong formula)
\item \textbf{B2}
\eng{I bet you \$50 the opposition loses the next election} (infelicitous without a response: \eng{OK – you're on}; Austin calls the required response uptake)
\item \textbf{C1}
\eng{Guilty as charged} (if accused known to be innocent by a jury member)
\item \textbf{C2}
\eng{I promise to come tomorrow} (if there is no intention to keep to the promise)
\end{itemize}


\myslide{Explicit and Implicit Performatives}
\begin{itemize}
\item \txx{Explicit Performatives}
  \begin{itemize}
  \item Tend to be first person
  \item The main verb  is a performative: 
    \lex{promise, warn, sentence, bet, pronounce, \ldots}
  \item You can use \lex{hereby}
  \end{itemize}
\item \txx{Implicit Performatives}
  \begin{exe}
    \ex \eng{You are hereby charged with treason} [by me]
    \ex \eng{Students are requested to be quiet in the halls} [by NTU]
    \ex \eng{10 bucks says they'll be late} [I bet you] 
    \ex \eng{Come up and see me some time!} [I invite you]
  \end{exe}
  Can be made explicit by adding an active perfomative verb
\end{itemize}

% \myslide{Elements of Speech Acts}
% \begin{description}
% \item \txx{Locutionary act} the act of saying something
% \item \txx{\underline{Illocutionary act}} the force of the statement 
% \item \txx{Perlocutionary act} the effects of the statement
% \end{description}
% Illocutionary force indicating devices(IFID)
% \begin{itemize}
% \item   word order
% \item    stress
% \item    intonation contour
% \item    punctuation
% \item    the mood of the verb
% \item     performative verbs: \eng{I (Vp) you that \ldots} 
% \end{itemize}

\myslide{Searle's speech act classification}
 \MyLogo{\cite{Searle:1969}}
  \begin{description}
  \item \txx{Declarative} changes the world (like performatives)
  \item \txx{Representative} describes the (speaker's view of the) world 
  \item \txx{Expressives}  express how the speaker feels
  \item \txx{Directives} get someone else to do something
  \item \txx{Comissives} commit oneself to a future action
  \end{description}

% \myslide{Felicity Conditions for Promising}
% \MyLogo{Searle (1969)}
 
% \begin{itemize}
% \item \textbf{Preparatory 1}: $H$ would prefer $S$'s doing $A$ to his
%   not doing $A$, and $S$ believes $H$ would prefer his doing $A$ to
%   his not doing $A$.
% \item \textbf{Preparatory 2:} It is not obvious to both $S$ and $H$
%   that $S$ will do $A$ in the normal course of events.
% \item \textbf{Propositional:} In expressing that $p$, $S$ predicates a
%   future act $A$ of $S$
% \item \textbf{Sincerity:}  $S$ intends to do $A$ 
% \item \textbf{Essential:} The utterance $e$ counts as an undertaking to do $A$
% \end{itemize}

%   \begin{tabular}{llll}
%     $S$ & Speaker & $A$ & Future Action \\
%     $H$ & Hearer  & $p$ & proposition expressed \\
%         &          & $e$ & linguistic expression  
%   \end{tabular}
\section{Indirect Speech Acts}


\myslide{Indirect speech acts}
\MyLogo{}
\begin{itemize}\addtolength{\itemsep}{-1ex}
\item 
 \begin{tabular}[t]{lcll}
   Sentence Type & & Speech Act & Example \\ \hline
  \txx{declarative}  &$\leftrightarrow$& \txx{assertion}  (statement) & 
  \eng{I sing.}\\
  \txx{interrogative} &$\leftrightarrow$& \txx{question}  &
  \eng{Do you sing?}   \\
  \txx{imperative} &$\leftrightarrow$& \txx{order} (request, command) &
  \eng{sing!} \\
  \txx{exclamative}&$\leftrightarrow$&  \txx{exclamation} & \eng{What a voice!}  \\
  \txx{optative} &$\leftrightarrow$& \txx{wish} & \eng{If only I could sing}
  \end{tabular}
  % \begin{description}
  % \item[statement] declarative:  \eng{I sing.}
  % \item[command] imperative: \eng{sing!}  
  % \item[question] interrogative: \eng{Do you sing?}  
  % \item[exclamation] exclamative: \eng{What a voice!}  
  % \end{description}
% \item Properties of Indirect Speech Acts:
%   \begin{itemize}
%   \item Multiplicity of meanings
%   \item Logical priority of meaning
%   \item Rationality
%   \item Conventionality
%   \item Politeness
%   \item Purposefulness
%   \end{itemize}
\end{itemize}  

\myslide{Literal and non-literal uses}

\begin{exe}
  \ex
  \begin{xlist}
    \ex \eng{Could you get that? }
    \ex \eng{Please pass the salt.}
  \end{xlist}
  \ex 
  \begin{xlist}
    \ex \eng{I wish you wouldn't do that.}
    \ex \eng{Please don't do that.}
  \end{xlist}
  \ex
  \begin{xlist}
    \ex \eng{You left the door open.}
    \ex \eng{Please close the door.}
  \end{xlist}
\end{exe}
\begin{itemize}
\item People have access to both the literal and non-literal meanings
\item Non literal meanings can be slower to understand
\item Some non-literal uses are very conventionalized 
  \\ \eng{Can/Could you X?} \into \eng{Please X}
\item Questioning the felicity conditions produces an indirect version
\end{itemize}


\myslide{Felicity Conditions for Requesting}
\MyLogo{Searle (1969), simplified}
These things must hold for an utterance to be a \txx{request}:
\begin{itemize}
\item \textbf{Preparatory 1}: $H$ (hearer) is able to perform  $A$ (future action)
\item \textbf{Preparatory 2}: It is not obvious that the $H$ would perform $A$  without being asked
\item \textbf{Propositional:} $S$ (speaker) predicates a future act $A$ of $H$
\item \textbf{Sincerity:}  $S$ wants $H$ to do $A$ 
\item \textbf{Essential:} The utterance $e$ counts as an attempt by $S$ to get $H$ to do $A$
\end{itemize}


\begin{itemize}
\item[?]  Form different indirect requests with the following strategies:\task
  \begin{enumerate}
  \item By querying the preparatory content of the direct request
  \item By stating the  propositional content of the direct request
  \item By querying the propositional content of the direct request
  \item By stating the sincerity condition of the direct request.
  \end{enumerate}

\end{itemize}




\myslide{Indirect Requests}

%\begin{small}
  \begin{itemize}
\item \textbf{Preparatory 1}: $H$ is able to perform  $A$
\item \textbf{Preparatory 2}: It is not obvious that the $H$ would perform $A$  without being asked
\item \textbf{Propositional:} $S$ predicates a future act $A$ of $H$
\item \textbf{Sincerity:}  $S$ wants $H$ to do $A$ 
\item \textbf{Essential:} The utterance $e$ counts as an attempt by $S$ to get $H$ to do $A$
\end{itemize}
%\end{small}
  \begin{itemize}
\item \textbf{Preparatory 1}: \eng{Can you tell me the time?}
\item \textbf{Preparatory 2}: \eng{Would you let me know the time?}
\item \textbf{Propositional:} \eng{Aren't you going to start your annotation?}
\item \textbf{Sincerity:}  \eng{I wish you would answer me}
%\item \textbf{Essential:} The utterance $e$ counts as an attempt by $S$ to get $H$ to do $A$
\end{itemize}


\section{Summary of Semantics and Pragmatics}

\myslide{The big picture}
\begin{itemize}
\item We can do many things with words
  \begin{itemize}
  \item Convey information
  \item Express attitudes
  \item Ask someone to do something
  \item Commit to doing something
  \item Change the state of the world (performatives)
  \end{itemize}
\newpage
\item We do this by building layers of inference (pragmatics) on top
  of our understanding of words and how they go together (semantics)
  \begin{itemize}
  \item Words have meanings, that can be described through semantic relations
  \item Words describe referents and situations, and can also show the
    speaker's attitudes
  \item Relations between participants in a situation are linked by semantic roles
  \item Sometimes word meaning is non-compositional, it comes from constructions
  \item The scope of reference can be changed by quantifiers and modification
  \end{itemize}
\item A skilled writer can use words to tell a story, \ldots
\end{itemize}

\myslide{What you have learned and are still learning}

\begin{itemize}
\item A gentle introduction to some of the basic semantic concepts
\item Some practical experience in analysis
  \begin{itemize}
  \item Word meaning (projects 1 and 2)
  \item Sentiment and connotation (project 1 and 2)
  \item Non-typical word meaning (project 3)
  \item How to define concepts (project 2 and 3)
  \item Identifying non-compositional expressions (project 3)
  \end{itemize}
% \item Holmes in popular culture,  Copyright and authorship
% \item Still to come
%   \begin{itemize}
%   \item Detective fiction 
%   \item Reading Sherlock Holmes
%   \item Watching Sherlock Holmes
%   \item Holmes in translation
%   \end{itemize}
\end{itemize}

\myslide{A polite request}

\begin{itemize}
\item I would like to ask permission to use the results of your
  analysis in projects 1--3.
\item This will help in further research into semantics, pragmatics
  and teaching.
\item Technically, that you release the data into \href{https://creativecommons.org/publicdomain/zero/1.0/}{the public domain}.
  \begin{itemize}
  \item You dedicate the work to the public domain by waiving all of
    your rights to the work worldwide under copyright law, including
    all related and neighboring rights, to the extent allowed by law.
  \item Anyone can copy, modify, distribute and use the data in
    performances, even for commercial purposes, all without asking
    permission.
  \end{itemize}
\item Your name will not be listed anywhere: I will thank the class
\item If you do not wish your data to be released, you can email me at
  any time until one week after the grades for the last project has
  been released and say you do not wish your data to be used.  This
  has no effect on your grade.
\end{itemize}

%%% FIXME
%%%
%%%  what did we learn
%%%
%%%  Further reading


%\section{Sentence Types}


%\myslide{Acknowledgments and References}

\small
\bibliographystyle{aclnat}
\bibliography{abb,mtg,nlp,ling}



\end{document}

%%% Local Variables: 
%%% coding: utf-8
%%% mode: latex
%%% TeX-PDF-mode: t
%%% TeX-engine: xetex
%%% End: 

