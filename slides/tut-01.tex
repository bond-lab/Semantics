\documentclass{article}

\title{Semantics: Tutorial One}
\author{Francis Bond}
\date{}%2011-08-15}
%\Restriction{}
%\rightfooter{}
%\leftheader{}
%\rightheader{}
\usepackage{mygb4e}
\newcommand{\lex}[1]{\textbf{\textit{#1}}}

%%% dynamically add answers
\input{answers}

\begin{document}
\maketitle

\begin{enumerate}
\item Try and define the following words without a dictionary:
  \begin{exe}
  \ex \lex{high school}
  \ex \lex{rat}
  \ex \lex{mouse}
  \ex \lex{copper}
  \ex \lex{dumpling}
\end{exe}

\ans{A: Any definition is ok, try to point out that
  \begin{itemize}
  \item \lex{high school} depends a lot on your country
  \item  \lex{rat}/\lex{mouse} are culturally divided, not scientifically
  \item  \lex{copper} in practice is used for non-pure metal, and was defined long before we knew about elements
  \item  \lex{dumpling} is also very culture specific
  \end{itemize}
 }

\item Try and define the translation equivalents of the above words in a language of your choice.
  \begin{quote}
    Have you encountered any difficulties?  If so what?
  \end{quote}

\ans{A: same as above}

\item Try to give some examples of demonstratives (like \lex{this} and \lex{that}) in another language. Look at
  least at words used to refer to things, locations and time 
  \begin{itemize}
  \item Do you get similar regularities?
  \end{itemize}


  \ans{A: Yes!}
\end{enumerate}

 \ans{Note: Please remind them to sign up for their groups!}
\end{document}
