\documentclass[a4paper]{article}

\title{Semantics: Tutorial Two}
\author{Francis Bond \url{<bond@ieee.org>}}
\date{}%2011-08-15}

%\newcommand{\ans}[1]{\hfill{#1}}
%\newcommand{\ans}[1]{}
%%% dynamically add answers
\input{answers}

\usepackage{multicol}
%\Restriction{}
%\rightfooter{}
%\leftheader{}
%\rightheader{}
\usepackage{mygb4e}
\newcommand{\lex}[1]{\textbf{\textit{#1}}}
\newcommand{\lx}[1]{\textbf{\textit{#1}}}
\newcommand{\ix}{\ex\it}
\newcommand{\con}[1]{\textsc{#1}}
\usepackage{url}
\usepackage[normalem]{ulem}
\newcommand{\ul}[1]{\uline{#1}}
\newcommand{\txx}[1]{\textbf{#1}}

\begin{document}
\maketitle

\begin{enumerate}
  \item Find a pair of absolute synonyms in a language that you speak.  
  \ans{sofa/couch (English)}

  \item Decide if the words in the following sets are absolute or 
near synonyms. How do you decide? What
type of criteria have you used?
\begin{exe}
  \ex \lex{tell, say, talk} \ans{near synonyms – differ in argument structure and context}
  \ex \lex{sad, unhappy} \ans{near synonyms – emotional nuance and usage frequency differ}
\end{exe}

\item  Classify the following pairs of opposites
\begin{multicols}{2}
\begin{exe}
  \ex \lex{temporary/permanent} \ans{simple}
  \ex \lex{red/green} \ans{taxonomic sister}
  \ex \lex{strong/weak} \ans{gradable}
  \ex \lex{open/shut} \ans{simple, reverse}
  \ex \lex{monarch/subject} \ans{converse}
  \ex \lex{advance/retreat} \ans{reverse}
  \ex \lex{buyer/seller} \ans{converse}
  \ex \lex{clean/dirty} \ans{gradable}
  \ex \lex{present/absent} \ans{simple}
  \ex \lex{yesterday/today} \ans{taxonomic sister}
\end{exe}
\end{multicols}

\item  Classify the following related words
\begin{multicols}{2}
\begin{exe}
  \ex \lex{begin/start} \ans{synonym}
  \ex \lex{class/education} \ans{domain}
  \ex \lex{slice/pie} \ans{portion-mass}
  \ex \lex{sand/grain} \ans{substance-element}
  \ex \lex{tree/forest} \ans{member-collection}
  \ex \lex{bee/swarm} \ans{member-collection}
  \ex \lex{palette/painting} \ans{domain}
  \ex \lex{segment/orange} \ans{portion-mass}
  \ex \lex{scalpel/surgery} \ans{domain}
  \ex \lex{vehicle/bicycle} \ans{hypernym}
  \ex \lex{smart/intelligent} \ans{synonym}
\end{exe}
\end{multicols}


\item  Below are some nouns ending in \lex{-er} and \lex{-or}. Using your intuition
about their meanings, discuss their status as agentive nouns. In
particular, are they derivable by regular rule or would they need to
be listed in the lexicon?  

\begin{quote}
  \lex{teacher, author, blazer, blinker, choker, loser, debtor, loner,
    mentor, reactor, roller, lecturer}
\end{quote}
Check your decisions against a dictionary's entries.

\abox{\lex{teacher}, \lex{lecturer}, \lex{roller} are regularly derived
  from verbs; \lex{reactor}, \lex{blinker}, \lex{choker}, \lex{loser}
  are derived with some meaning shift; \lex{author}, \lex{blazer},
  \lex{debtor}, \lex{loner} and \lex{mentor} are lexicalized and must
  be listed individually.  Note some of these words can be ambiguous, make the students define them as you discuss them}

\end{enumerate}

If there is more time, try to come up with your own examples of the different meaning alternations.

\vfill
\paragraph{Acknowledgments} Some of these questions are partially
based on exercises from Saeed (2003, ch3)
\end{document}
