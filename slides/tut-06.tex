\documentclass[a4paper]{article}

\title{Semantics: Tutorial Four — Implicature, Politeness, and Scalar Meaning}
\author{Francis Bond \url{<bond@ieee.org>}}
\date{}%2011-08-15}
\usepackage{fontenc}
\usepackage{polyglossia}
\setmainlanguage{english}
\setmainfont{TeX Gyre Pagella}
\setsansfont[Ligatures=TeX]{TeX Gyre Heros}
\usepackage{xeCJK}
\setCJKmainfont{Noto Sans CJK JP}
\setCJKsansfont{Noto Sans CJK JP}
\setCJKmonofont{Noto Sans CJK JP}
%\newcommand{\ans}[1]{\hfill{#1}}
%\newcommand{\ans}[1]{}
%%% dynamically add answers
\input{answers}

\usepackage[colorlinks]{hyperref}
\usepackage{xspace}
\newcommand{\ent}{\ensuremath{\Rightarrow}\xspace}
\newcommand{\nent}{\ensuremath{\not\Rightarrow}\xspace}
\newcommand{\Y}[1]{\textbf{#1}}

\usepackage{multicol}
%\Restriction{}
%\rightfooter{}
%\leftheader{}
%\rightheader{}
\usepackage{mygb4e}
\newcommand{\lex}[1]{\textbf{\textit{#1}}}
\newcommand{\lx}[1]{\textbf{\textit{#1}}}
\newcommand{\ix}{\ex\it}
\newcommand{\eng}{\textit}
\newcommand{\con}[1]{\textsc{#1}}
\usepackage{url}
\usepackage[normalem]{ulem}
\newcommand{\ul}[1]{\uline{#1}}
\newcommand{\txx}[1]{\textbf{#1}}

\begin{document}
\maketitle

\noindent In this tutorial, we'll examine how meaning goes beyond literal content — including implicatures, politeness strategies, and face-threatening acts.

%\item Make your own joke using strict/sloppy identity readings.
\begin{enumerate}
\item Hedges are used when you know you will flout a maxim.  Which
  maxim is flouted in the following hedges (and why)?:

\begin{exe}

\ex \textit{This may be a bit confusing, but I remember being in a car.}
\abox{Manner}

\ex \textit{I may be mistaken, but I thought I saw a wedding ring on her finger.}
\abox{Quality}
%\ex \textit{ I'm not sure if this is right, but I heard it was a secret ceremony in Hawaii.
%\ex \textit{ He couldn't live without her, I guess.


%\ex \textit{  So, to cut a long story short, we grabbed our stuff and ran.
\ex \textit{I won't bore you with all the details, but it was an exciting trip.}
\abox{Quantity}


\ex \textit{I don't know if this is important, but some of the files are missing.}
\abox{Relation}

\ex \textit{As far as I know, they're married.}
\abox{Quality}

\ex \textit{This may sound like a dumb question, but whose handwriting is this?}
\abox{Relation}
% \ex \textit{  Not to change the subject, but is this related to the budget?


\ex \textit{I don't know if this is clear at all, but I think the other car was reversing.}
\abox{Manner}

\ex \textit{As you probably know, I am afraid of dogs.}
\abox{Quantity}
\end{exe}
\item Consider this sentence: “John \textbf{believes} that aliens exist.”  
\\ Why might it trigger an implicature based on a Horn scale?
  \abox{\textit{Believe} is part of a scale like \textit{〈think, believe, know, be certain〉}.
  Saying “believe” implicates that John does not “know” or is not “certain” — because those would be stronger, more informative choices.
  Scalar implicatures here involve epistemic strength: the speaker signals uncertainty by choosing a weaker term.}

\item What is the scalar implicature in this exchange?
  \begin{description}
    \item[Boss:] Did you finish the report?
    \item[Employee:] I \textbf{tried} to.
  \end{description}
  \abox{``Tried'' belongs to a scale like \textit{〈tried, began, succeeded〉}.
  Choosing ``tried'' implicates that the stronger outcomes (like “succeeded”) didn’t happen.
   This implies failure or at least incomplete success, even though the speaker avoids saying so directly.}

\item For the next questions on politeness, (i) answer the question and (ii) translate each example into a language you speak, and see if you would use a similar strategy.
\begin{enumerate}

\item Which face is threatened in this interaction?
  \begin{description}
    \item[Boss:] I need that report on my desk in one hour.
  \end{description}
  \abox{This threatens the employee’s \textbf{negative face} — their freedom to act is restricted by an imposition. 
\\  It’s a indirect request which softens it a little.}
 \item Which face is threatened in this interaction?

 \begin{description}
    \item[Peer:] Actually, that’s not how it works — I think you misunderstood the instructions.
  \end{description}
  \abox{This threatens the hearer’s \textbf{positive face} — their self-image and competence are being challenged.
\\  The use of \textit{I think} softens it a little.}

\item What kind of face is appealed to here?
  \begin{description}
    \item[Friend:] You’re so good at design stuff — can you help me make a poster?
  \end{description}
  \abox{This appeals to the hearer’s \textbf{positive face} — their desire to be liked and appreciated.
  Complimenting their skills softens the upcoming request.}

\item How does this speaker preserve the hearer’s face?
  \begin{description}
    \item[Colleague:] I’m sorry to bother you, but could you maybe take a look at this when you have a chance?
  \end{description}
  \abox{This protects the hearer’s \textbf{negative face} by minimizing imposition. 
  The speaker uses hedges (“maybe”), indirectness, and deference (“I’m sorry to bother you”) to be polite.}


\item What politeness strategy is used in this refusal?
  \begin{description}
    \item[Student:] Could you write me a recommendation letter?
    \item[Lecturer:] I’d love to help, but I’m afraid I’m already overloaded with deadlines this month.
  \end{description}
  \abox{This is a face-saving refusal that mitigates the threat to the student's \textbf{positive face}.
  It avoids a bald “no” and uses a reason to show regret and respect.}


\item How does this strategy protect both positive and negative face?
  \begin{description}
    \item[Email:] I hope this finds you well. If it's not too much trouble, would you mind reviewing my draft by Friday?
  \end{description}
  \abox{The writer attends to the recipient’s \textbf{positive face} (friendly greeting) and \textbf{negative face} (minimizing imposition with indirect phrasing).
  It’s a classic example of polite written request.}

\end{enumerate}

\item Explain what is going on \href{https://www.youtube.com/watch?v=lLR-V2S0DC8}{here}:
  \\ (from \textit{The Pink Panther Strikes Again (1976))}
  \begin{description}
    \item[Clouseau:] Does your dog bite?
    \item[Hotel Clerk:] No.
    \item[Clouseau:] [bowing down to pet the dog] Nice doggie.
      \\ {}[Dog barks and bites Clouseau in the hand]
    \item[Clouseau:] I thought you said your dog did not bite!
    \item[Hotel Clerk:] That is not my dog. 
    \end{description}
 \abox{The clerk violates Relation — he answers literally, but not relevantly. Clouseau meant “the dog next to you,” not just any dog.}

\item Rephrase the following commands into indirect requests, first in English, then in another language:
  \begin{enumerate}
  \item Tell me the time

    \abox{
    \begin{itemize}
      \item Could you tell me what time it is?
      \item Do you happen to know the time?
      \item I was wondering if you might have the time?
    \end{itemize}
    }

  \item Lend me money

    \abox{
    \begin{itemize}
      \item Would it be possible for you to lend me some money?
      \item Do you think you could help me out with a loan?
      \item I hate to ask, but is there any chance I could borrow a bit of money?
    \end{itemize}
    }

  \item Give me your phone number

    \abox{
    \begin{itemize}
      \item Could I have your phone number, if you don’t mind?
      \item Would you be okay sharing your number with me?
      \item Is it alright if I get your number?
    \end{itemize}
    }

  \item Leave me alone

    \abox{
    \begin{itemize}
      \item Would you mind giving me some space for a bit?
      \item I need some time to myself right now — could we talk later?
      \item I’d appreciate it if I could be alone for a while.
    \end{itemize}
    }
  \end{enumerate}

\end{enumerate}


\end{document}

%%% Local Variables: 
%%% coding: utf-8
%%% mode: latex
%%% TeX-PDF-mode: t
%%% TeX-engine: xetex
%%% End: 
