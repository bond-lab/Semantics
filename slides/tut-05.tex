\documentclass[a4paper]{article}

\title{Semantics: Tutorial Five}
\author{Francis Bond \url{<bond@ieee.org>}}
\date{}%2011-08-15}
\usepackage{fontenc}
\usepackage{polyglossia}
\setmainlanguage{english}
\setmainfont{TeX Gyre Pagella}
\setsansfont[Ligatures=TeX]{TeX Gyre Heros}
\usepackage{xeCJK}
\setCJKmainfont{Noto Sans CJK JP}
\setCJKsansfont{Noto Sans CJK JP}
\setCJKmonofont{Noto Sans CJK JP}
%\newcommand{\ans}[1]{\hfill{#1}}
%\newcommand{\ans}[1]{}
%%% dynamically add answers
\input{answers}

\usepackage{xspace}
\newcommand{\ent}{\ensuremath{\Rightarrow}\xspace}
\newcommand{\nent}{\ensuremath{\not\Rightarrow}\xspace}
\newcommand{\Y}[1]{\textbf{#1}}

\usepackage{multicol}
%\Restriction{}
%\rightfooter{}
%\leftheader{}
%\rightheader{}
\usepackage{mygb4e}
\usepackage[e,j]{mtg2e}
\newcommand{\lex}[1]{\textbf{\textit{#1}}}
\newcommand{\lx}[1]{\textbf{\textit{#1}}}
\newcommand{\ix}{\ex\it}
%\newcommand{\eng}{\textit}
\newcommand{\con}[1]{\textsc{#1}}
\usepackage{url}
\usepackage[normalem]{ulem}
\newcommand{\ul}[1]{\uline{#1}}
\newcommand{\txx}[1]{\textbf{#1}}

\begin{document}
\maketitle
\begin{enumerate}
  \item Consider the following English phrases:
    \begin{enumerate}
    \item walk quickly
      \abox{compositional}
    \item spill the tea
      \abox{non-compositional:  ``reveal gossip or secrets''}
    \item green thumb
      \abox{non-compositional: ``talent for gardening''}
    \item write an email
      \abox{compositional}
    \item night sky
      \abox{compositional}
    \item break the ice
      \abox{non-compositional: ``to initiate social interaction, reduce tension''}
    \end{enumerate}
    Which ones have a non-compositional reading, and what is it?
    \\  e.g. \textit{hit the sack} --- non-compositional: ``go to bed'';\\
    \textit{hit the nail} --- compositional
    
\item Give an example of a multiword expression meaning the following in a language you speak.
  \begin{exe}
    \ex become angry
    \abox{\jpn{hara ga tatsu} (腹が立つ) ``my stomach rises''}
    \ex die
    \abox{\eng{pass away}}
    \ex get up early
    \abox{rise with the lark
    \\ \eng{Vstávat se slepicemi} ``get up with the chickens''}
  \ex do two things at once
  \abox{ \eng{Zabít dvě mouchy jednou ranou} ``kill two flies with one hit''
\\ kill two birds with one stone.}
\ex rain heavily
\abox{\jpn{doshaburi} 土砂降り ``rain eath and sand''
  \\ \eng{rain cats and dogs}
  \\ \eng{Lije jako z konve}  ``raining as if from a watering can''}
  \end{exe}
  
\item Lakoff and Johnson (1980) proposed the following image schemas for love:
  \begin{itemize}
  \item LOVE IS A JOURNEY
  \item LOVE IS A FORCE (electromagnetic, gravitational)
  \item LOVE IS WAR
  \end{itemize}
Organise the following metaphors into the above three schemas.
\begin{exe}
  \ex \textit{They lost their momentum}
  \abox{LOVE is a FORCE}
 \ex \textit{There were sparks between us}
  \abox{LOVE is a FORCE}
  \ex \textit{Look how far we've come}
  \abox{LOVE is a JOURNEY}
\ex \textit{We're at a crossroad}
  \abox{LOVE is a JOURNEY}
  \ex \textit{He overpowered her}
  \abox{LOVE is WAR}
\ex \textit{I could feel the electricity between them}
  \abox{LOVE is a FORCE}
\ex \textit{We'll just have to go our separate ways}
  \abox{LOVE is a JOURNEY}
\ex \textit{We can't turn back now}
  \abox{LOVE is a JOURNEY}
\ex \textit{His whole life revolves around her}
  \abox{LOVE is a FORCE}
\ex \textit{She is besieged by suitors}
  \abox{LOVE is WAR}
\ex \textit{They are uncontrollably attracted to each other}
  \abox{LOVE is a FORCE}
\ex \textit{He is known for his conquests}
  \abox{LOVE is WAR}
\end{exe}
Can you think of other metaphors that do not fit into the above three schemas?

\item For any two languages that you know, discuss similarities and
  differences in conventionalized metaphors of body parts
  (e.g. \eng{hand of a watch}, \jpn[well-known (lit: face is wide)]{kao-ga hiroi}).
 \abox{\begin{itemize}
  \item \jpn{atama ga katai} (頭がかたい) ``stubborn --- head is hard''
    \\ HEAD is MIND; THOUGHT is OBJECT
  \item \jpn{kuchi ga karui} (口が軽い) ``loose lipped --- mouth is
    light''
    \\ can't keep a secret
    \\ MOUTH is VOICE; THOUGHT is OBJECT
  \item \jpn{kuchi ga katai} (口がかたい) ``close mouthed --  mouth is
    hard''
    \\ can keep a secret
    \\ MOUTH is VOICE; THOUGHT is OBJECT
  \end{itemize}}
 
\end{enumerate}
\end{document}

%%% Local Variables: 
%%% coding: utf-8
%%% mode: latex
%%% TeX-PDF-mode: t
%%% TeX-engine: xetex
%%% End: 
